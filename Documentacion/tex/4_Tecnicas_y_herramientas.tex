\capitulo{4}{Técnicas y herramientas}



\tablaSmall{Manejadores de paquetes y Corredeores de tareas utilizados en el proyecto}{|l|}{manejadores}
{ Manejadores de paquetes y Corredeores de tareas\\ }{ 
Bower \\
Grunt \\
}

\tablaSmall{Formato de intercambio de datos}{|l|}{intercambiodedatos}
{ Formato de intercambio de datos\\ }{ 
JSON\\
}

\tablaSmall{Lenguajes y frameworks utilizados en el proyecto}{|l|}{lenguajes}
{ Lenguajes y Frameworks\\ }{ 
JavaScript \\
Python\\
AngularJS\\
Java (Selenium)\\
Node.js\\
}

\tablaSmall{Herramientas utilizadas para realizar los test del proyecto}{|l|}{pruebas}
{ Herramientas para pruebas \\ }{ 
Selenium \\
Webserver Stress Tool \\
}

\tablaSmall{Software de control de versiones}{|l|}{controlversion}
{ Control de versiones\\ }{ 
Git\\
ZenHub\\
}

\tablaSmall{Documentación}{|l|}{documentacion}
{ Documentación\\ }{ 
\TeX{}\\
Zotero \\
}

\tablaSmall{Servicios web}{|l|}{servicios}
{ Servicios web\\ }{ 
Heroku\\
Firebase\\
}

\tablaSmall{Herramientas utilizadas para la interfaz del proyecto}{|l|}{interfaz}
{ Interfaz \\ }{ 
HTML5 \\
CSS3 \\
}

\section{HTML5}\label{HTML5}

HTML5 es un lenguaje de marcado que se ha utilizado para mostrar al usuario la información así como para aprovechar algunas de sus caractrísticas para el procesamiento de información.

Se ha utilizado porque es muy utilizado, y es soportado por todos los dispositivos.

\section{CSS3}\label{CSS3}

CSS3 lenguaje de hojas de estilo en cascada utilizado para dar una impresión diferente al código HTML, hay que decir que permite crear estilos de forma rápida y sencilla, pero al aumentar el número de clases y el anidamiento es muy difícil de mantener. 

\section{JavaScript}\label{JavaScript}

JavaScript es un lenguaje de scripting, que en este proyecto es ampliamente utilizado en todos los ámbitos, ya sea para cambiar el flujo de salida (HTML,) como para procesamiento de información y llamadas a un sevidor. Ha sido elegido por su versatilidad y capacidad de adaptarse a cualquier entorno.

\section{AngularJS}\label{AngularJS}

AngularJS (Google) es un framework JavaScript de desarrollo de aplicaciones web en el lado cliente, y utiliza el patrón MVC (Model-View-Controller).

AngularJS adquiere las buenas características que tiene JavaScript, pero su característica mas importante es que permite realizar un gran control sobre el flujo de salida.

\section{Bower}\label{Bower}

Bower es un gestor de paquetes para la web, es indispensable, ya que permite manejar los paquetes del proyecto de una forma simple y rápida.

\section{Grunt}\label{Grunt}

Grunt es un corredor de tareas de javascript, que es utilizado para centralizar tareas y scripts, para tener que evitar procedimientos repetitivos. Si se une con Bower permite trabajar con una gran cantidad de paquetes y scripts de forma sencilla. 

\section{JSON}\label{JSON}
JSON (JavaScript Object Notation) es un formato de intercambio de datos. Es simple tanto a  la  hora de leerlo, como de escribirlo, es ideal para la transmisión de objetos por la red, y es esa la razón por la que es utilizado.

\section{Node.js}\label{Node.js}

Node es un intérprete Javascript en el lado del servidor. Permite construir aplicaciones muy escalables y  maneja miles de conexiones simultáneas en una sólo una máquina física.

En el proyecto ha sido utilizado para crear la Rest API para manejar la base de datos, se ha elegido por su simpleza, aunque cabe destacar que debido a que te permite manejar muchas opciones, es fácil equivocarse y realizar acciones que no se tenían pensadas.

\section{python}\label{python}

Python es un lenguaje de programación interpretado que busca ser un código legible. Es un lenguaje de programación multiparadigma, programación imperativa y programación funcional.

En este proyecto ha sido utilizado de manera escasa, pero ha sido muy útil para realizar pruebas individuales. 

\section{Git}\label{Git}

Git es un software de control de versiones diseñado por Linus Torvalds, se ha elegido y usado porque es usado por el sevicio web heroku así como por GitHub.

\section{Latex}\label{Latex}

\TeX{} y Latex han sido utilizados para documentar ya que a pesar de que al principio se hace un poco difícil trabajar con ello, los resultados son excelentes.

\section{Heroku}\label{Heroku}

Heroku es una plataforma de servicio de computación en la Nube que soporta distintos lenguajes de programación.
En el proyecto ha sido utilizado para sostener es servidor con Node.js con la Rest API. Se ha elegido Heroku por recomendación del tutor, asi como su facilidades para comenzar en ello, ya que ofrece servicios gratuitos y muy buenos tutoriales.

\section{Firebase}\label{Firebase}

Firebase es una plataforma de desarrollo de aplicaciones móviles y web, es un sevicio ofrecido por google y permite tanto el hosting, base de datos, almacenamiento. Ha sido elegido en este proyecto por su base da datos (JSON) no SQL en tiempo real, que permite trabajar con la base de datos de forma muy rápida y sencilla. El utilizar una base de datos JSON al igual que AnyPlace (aplicación en la que se basa Geoindoor), si se quisiera aunar bases de datos, se conseguiría de una forma relativamente sencilla.


