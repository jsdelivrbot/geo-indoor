\apendice{Especificación de Requisitos}

\section{Introducción}
Este anexo recoge la especificación de requisitos, la cual define el comportamiento del sistema desarrollado, es utilizado tanto por el cliente y por el equipo para tener una concepto generalizado de lo que se quiere, para que ambas partes lleguen a una idea común. Para este fin muchas veces son utilizados diagramas los cuales aclaren el concepto y lo que se necesita.

Podemos distinguir entre requisitos funcionales y requisitos no funcionales.
\begin{itemize}
\item
\textbf{Requisito funcional:} es aquel que especifica una función o un servicio que debe cumplir un sistema.
\item
\textbf{Requisito no funcional:} este especifica restricciones sobre el diseño y la implementción del sistema.
\end{itemize}

Las características de una buena especificación de requisitos son definidas por:
\begin{itemize}
\item
\textbf{Completa:} deben estar todos los requerimientos y todas sus relaciones.
\item
\textbf{Consistente:} todos los requerimientos y otros documentos de especificación se debe relacionar de forma coherente.
\item
\textbf{Inequívoca:} se debe ser claro y darse a entender.
\item
\textbf{Correcta:} el sistema debe cumplir con todos los requisitos especificados.
\item
\textbf{Trazable:} los requerimientos deben de estar ordenados y organizados de una manera tal, que sea sencilla su identificación.
\item
\textbf{Priorizable:} los requisitos deben estar organizados de forma jerárquica, según su importancia.
\item
\textbf{Modificable:} los requerimientos deben ser fácilmente modificables.
\item
\textbf{Verificable:} se debe poder probar todo requerimiento.
\end{itemize}


\section{Objetivos generales}
\begin{itemize}
\item
Desarrollar un sistema que permita la localización del individuo en interiores.
\item
Desarrollar un sistema que permita la localización de lugares en interiores.
\item
Desarrollar un sistema que permita el trazado de rutas en interiores.
\item
Desarrollar un sistema que permita mostrar lugares y rutas predefinidas en interiores.
\end{itemize}
\section{Catalogo de requisitos}
A partir de los objetivos anteriormente citados obtenemos los siguientes requisitos.

\subsection{Requisitos funcionales}

\begin{itemize}
\item
\textbf{RF-1 Gestión de edificios:} la herramienta debe permitir la gestión de un edifico.
\begin{itemize}
	\item
	\textbf{RF-1.1 Emplazar edificio:} la herramienta debe permitir la colocación de un edifico sobre el plano de localización
	\item
	\textbf{RF-1.2 Detallar información del edificio:} se debe permitir adjuntar información al edificio (nombre, descripción, etc\ldots).
	\item
	\textbf{RF-1.3 Emplazar plano:} la herramienta debe permitir la colocación del plano del edifico sobre el plano de localización.
	\item
	\textbf{RF-1.4 Redimensionar y mover el plano:} se debe permitir el movimiento y redimensión del plano añadido.
	\item
	\textbf{RF-1.5 Borrar edifico:} se debe permitir el borrado de un edificio.
\end{itemize}
\item
\textbf{RF-2 Gestión de pois:} la herramienta debe permitir la gestión de pois.
\begin{itemize}
	\item
	\textbf{RF-2.1 Emplazamiento de pois:} se debe permitir la colocación de puntos de interés sobre el plano.
	\item
	\textbf{RF-2.2 Detallar información del poi:} se debe permitir adjuntar información al poi (nombre, descripción, tipo, etc\ldots).
	\item
	\textbf{RF-2.3 Ubicar poi:} la herramienta debe permitir la colocación del poi sobre una localización del plano de un edifico.
	\item
	\textbf{RF-2.4 Enlazar pois:} se debe permitir crear un camino entre pois.
\end{itemize}
\item
\textbf{RF-3 Gestión de rutas:} se debe permitir la gestión de rutas predefinidas.
\begin{itemize}
	\item
	\textbf{RF-3.1 Crear ruta:} se debe permitir crear una ruta predefinida o predeterminada.
	\item
	\textbf{RF-3.2 Detallar información de la ruta:} se debe permitir agregar información a la ruta, como el nombre o la planta (el número de planta se añade de forma automática, lo que nos permite hacer rutas multinivel).
	\item
	\textbf{RF-3.3 Añadir pois a ruta:} la herramienta debe se capaz de añadir puntos de interés a la ruta.
	\item
	\textbf{RF-3.4 Trazar ruta:} la herramienta tiene que ser capaz de mostrar el trazado de la ruta.
	\item
	\textbf{RF-3.5 Seleccionar ruta:} la herramienta debe permitir elegir entre las diferentes rutas existentes.
	\item
	\textbf{RF-3.6 Limpiar ruta:} la herramienta debe permitir dejar de mostrar el trazado de la ruta.
	\item
	\textbf{RF-3.7 Borrar ruta:} la herramienta debe permitir el borrado de una ruta elegida.
\end{itemize}
\item
\textbf{RF-4 Visualización de edificios:} el edificio debe estar disponible para todos los usuarios si el usuario administrador o creador así lo decide.
\begin{itemize}
	\item
	\textbf{RF-4.1 Buscar edificio:} la herramienta debe permitir que el edificio se pueda encontrar y visualizar a partir de su identificador o nombre.
	\item
	\textbf{RF-4.2 Visualizar información:} la herramienta debe permitir que la información del edificio sea visible a los usuarios visitantes, por lo tanto se debe visualizar también el plano.
\end{itemize}
\item
\textbf{RF-5 Visualización de pois:} los pois de un edificio deben ser visibles
\begin{itemize}
	\item
	\textbf{RF-5.1 Visualizar información:} la herramienta debe permitir que la información del poi sea visible a los usuarios visitantes.
	\item
	\textbf{RF-5.2 Marcar un poi:} la herramienta debe permitir a los usuarios marcar o seleccionar un poi.
	\item
	\textbf{RF-5.3 Proyectar camino de forma automática:} la herramienta debe permitir que al seleccionar un poi se pueda dibujar un camino automático desde la ubicación del usuario hasta el poi indicado.
\end{itemize}
\item
\textbf{RF-6 Visualización de rutas:} la herramienta debe permitir a los usuarios visualizar las rutas predefinidas del edificio seleccionado
\begin{itemize}
	\item
	\textbf{RF-6.1 Seleccionar ruta:} la herramienta debe permitir seleccionar rutas definidas del edificio.
	\item
	\textbf{RF-6.2 Mostrar ruta:} la herramienta debe permitir mostrar el trazado de la ruta.
	\item
	\textbf{RF-6.3 Limpiar ruta:} la herramienta debe permitir dejar de mostrar el trazado de la ruta.
\end{itemize}
\end{itemize}

\subsection{Requisitos no funcionales}

\begin{itemize}
	\item
	\textbf{RF-1 Intuitiva:} la herramienta debe ser intuitiva y fácil de entender para que así el usuario se acostumbre fácilmente a su uso.
	\item
	\textbf{RF-2 Usable:} la herramienta debe ser fácil de usar, para atraer tanto a gente con voluntad de aprender como no.
	\item
	\textbf{RF-3 Portable:} la herramienta debe de necesitar de instalación alguna más allá de sus dependencias lógicas (Python y navegador).
	\item
	\textbf{RF-4 Rápida:} la herramienta debe proporcionar un servicio rápido, y con resultados tempranos para que el usuario este a gusto trabajando desde el primer instante.
	\item
	\textbf{RF-5 Para todo el mundo:} la herramienta debe tener como final, que un gran espectro de la población la use para ser así mas útil para la sociedad.
\end{itemize}

\section{Especificación de requisitos}

\subsection{Actores}

El sistema diferencia entre dos tipos de actores.

\begin{itemize}
	\item
	\textbf{Usuario de Architect:} este usuario debe de estar autenticado para poder trabajar con la herramienta, se encarga de producir información (crear edificios, rutas, pois, etc\ldots) que será vista por los usuarios de Viewer.
	\item
	\textbf{Usuario de Viewer:} este usuario no debe de estar autenticado para poder trabajar con la herramienta, y no se encarga de producir información, si no de su visualización.
\end{itemize}

\subsection{Aclaraciones}

\begin{itemize}
	\item
	\textbf{Geoindoor:} herramienta o aplicación web que esta formada por dos sub aplicaciones, Architect y Viewer.
	\item
	\textbf{Architect:} herramienta o aplicación web que es utilizada por los usuarios para introducir y editar información sobre los edificios, que luego serán disponibles por los usuarios del Viewer.
	\item
	\textbf{Viewer:} herramienta o aplicación web que es utilizada por los usuarios para visualizar la información sobre los edificios (rutas, pois, etc\ldots) introducidos a través del Architect.
\end{itemize}

\newpage
\subsection{Diagrama de casos de uso}

\imagenResize{0.65}{img/GestionDeEdificio}{Gestión de edificio}{GestionDeEdificio}
\newpage
\imagenResize{0.65}{img/GestionDePois}{Gestión de pois}{GestionDePois}
\newpage
\imagenResize{0.55}{img/GestionDeRutas}{Gestión de rutas}{GestionDeRutas}
\newpage
\imagenResize{0.5}{img/VisualizDeEdificios}{Visualización de edificios}{VisualizDeEdificios}
\imagenResize{0.5}{img/VisualizDePois}{Visualización de pois}{VisualizDePois}
\newpage
\imagenResize{0.55}{img/VisualizDeRutas}{Visualización de rutas}{VisualizDeRutas}

\newpage
\subsubsection{Diagrama de caso de uso (General)}

\imagenResize{0.25}{img/DiagramaCasoDeUso}{Diagrama de caso de uso}{DiagramaCasoDeUso}


\newpage
\subsection{Casos de uso}

\tablaSinColores{CU-01 Gestión de edificio}
{L{3.5cm} L{10cm}}
{2}
{Tabla CU-01}
{\textbf{CU-01} & \textbf{Gestión de edificio} \\}
{\textbf{Versión} 				& 1.0\\ 
	\textbf{Autor} 				& Juan Pedro Pascual Vitores\\
	\textbf{Requisitos asociados} 	& RF-1, RF-1.2, RF-1.3, RF-1.4, RF-1.5\\
	\textbf{Descripción} 			& 
	Permite al usuario la administración y gestión de un edifico.\\
	\textbf{Precondiciones} 		& 
	\begin{itemize}
		\item Se encuentran disponibles las bases de datos.
		\item Se encuentran disponibles las REST API.
		\item Se accede como usuario Architect (autenticado).
	\end{itemize}
	\\
	\textbf{Acciones} 				& 
	\begin{enumerate}
		\item El usuario accede a la herramienta Architect.
		\item El usuario se autentica.
		\item El usuario interacciona con la herramienta para introducir la información concerniente a un edificio.
		\item Se introduce la información.
	\end{enumerate}
	\\
	
	\textbf{Postcondiciones} 		& 
	\begin{itemize}
		\item La información se almacena en la base de datos.
	\end{itemize}
	\\
	\textbf{Excepciones} 			& 
	\begin{itemize}
		\item La bases de datos no están disponible.
		\item Las REST API no están disponibles.
		\item La información introducida no es válida
	\end{itemize}
	
	\\
	\textbf{Importancia} 			& Alta\\}

%%%%%%%%%%%%%%%%%%%%%%%%%%%%%%%%%%%%%%%%%%%%%%%%%%%%%%%%%%%%%%%%%%%%%%%%%%%%%%%%%%%%%%%%%%%%%%%%%%%%%%%%%%%%%%%%%%%%%%%%%%%%%

\tablaSinColores{CU-02 Emplazar edificio}
{L{3.5cm} L{10cm}}
{2}
{Tabla CU-02}
{\textbf{CU-02} & \textbf{Emplazar edificio} \\}
{\textbf{Versión} 				& 1.0\\ 
	\textbf{Autor} 				& Juan Pedro Pascual Vitores\\
	\textbf{Requisitos asociados} 	& RF-1.1\\
	\textbf{Descripción} 			& 
	Permite al usuario la colocación de un edifico sobre el plano de localización.\\
	\textbf{Precondiciones} 		& 
	\begin{itemize}
		\item Se encuentran disponibles las bases de datos.
		\item Se encuentran disponibles las REST API.
		\item Se accede como usuario Architect (autenticado).
	\end{itemize}
	\\
	\textbf{Acciones} 				& 
	\begin{enumerate}
		\item El usuario accede a la herramienta Architect.
		\item El usuario se autentica.
		\item El usuario elige una posición en el plano.
		\item El usuario pincha en la pestaña ``Buildings" de su panel de control.
		\item El usuario pincha en el botón ``Add'', después en el icono del edificio y arrastra el puntero hasta una posición del plano.
		\item El usuario pulsa sobre el botón ``Add" del panel emergente, para añadir de forma definitiva el edificio.
	\end{enumerate}
	\\
	
	\textbf{Postcondiciones} 		& 
	\begin{itemize}
		\item La información sobre el edificio se almacena en la base de datos.
		\item La información posteriormente será visible.
	\end{itemize}
	\\
	\textbf{Excepciones} 			& 
	\begin{itemize}
		\item La bases de datos no están disponibles.
		\item Las REST API no están disponibles.
		\item La información introducida no es válida
	\end{itemize}
	
	\\
	\textbf{Importancia} 			& Alta\\}

%%%%%%%%%%%%%%%%%%%%%%%%%%%%%%%%%%%%%%%%%%%%%%%%%%%%%%%%%%%%%%%%%%%%%%%%%%%%%%%%%%%%%%%%%%%%%%%%%%%%%%%%%%%%%%%%%%%%%%%%%%%%%

\tablaSinColores{CU-03 Detallar información de edificio}
{L{3.5cm} L{10cm}}
{2}
{Tabla CU-03}
{\textbf{CU-03} & \textbf{Detallar información de edificio} \\}
{\textbf{Versión} 				& 1.0\\ 
	\textbf{Autor} 				& Juan Pedro Pascual Vitores\\
	\textbf{Requisitos asociados} 	& RF-1.2\\
	\textbf{Descripción} 			& 
	Permite al usuario adjuntar información al edificio (nombre, descripción, etc\ldots).\\
	\textbf{Precondiciones} 		& 
	\begin{itemize}
		\item Se encuentran disponibles las bases de datos.
		\item Se encuentran disponibles las REST API.
		\item Se accede como usuario Architect (autenticado).
	\end{itemize}
	\\
	\textbf{Acciones} 				& 
	\begin{enumerate}
		\item El usuario accede a la herramienta Architect.
		\item El usuario se autentica.
		\item El usuario elige una posición en el plano.
		\item El usuario pincha en la pestaña ``Buildings" de su panel de control.
		\item El usuario pincha en el botón ``Add'', después en el icono del edificio y arrastra el puntero hasta una posición del plano.
		\item El usuario rellena el formulario del panel emergente.
		\item El usuario pulsa sobre el botón ``Add" del panel emergente, para añadir de forma definitiva el edificio.
	\end{enumerate}
	\\
	
	\textbf{Postcondiciones} 		& 
	\begin{itemize}
		\item La información sobre el edificio se almacena en la base de datos.
		\item La información posteriormente será visible.
	\end{itemize}
	\\
	\textbf{Excepciones} 			& 
	\begin{itemize}
		\item Las bases de datos no está disponibles.
		\item Las REST API no están disponibles.
		\item La información introducida no es válida
	\end{itemize}
	
	\\
	\textbf{Importancia} 			& Alta\\}

%%%%%%%%%%%%%%%%%%%%%%%%%%%%%%%%%%%%%%%%%%%%%%%%%%%%%%%%%%%%%%%%%%%%%%%%%%%%%%%%%%%%%%%%%%%%%%%%%%%%%%%%%%%%%%%%%%%%%%%%%%%%%

\tablaSinColores{CU-04 Emplazar plano}
{L{3.5cm} L{10cm}}
{2}
{Tabla CU-04}
{\textbf{CU-04} & \textbf{Emplazar plano} \\}
{\textbf{Versión} 				& 1.0\\ 
	\textbf{Autor} 				& Juan Pedro Pascual Vitores\\
	\textbf{Requisitos asociados} 	& RF-1.3\\
	\textbf{Descripción} 			& 
	Permite al usuario la colocación del plano del edificio sobre el plano de localización.\\
	\textbf{Precondiciones} 		& 
	\begin{itemize}
		\item Se encuentran disponibles las bases de datos.
		\item Se encuentran disponibles las REST APIs.
		\item Se accede como usuario Architect (autenticado).
		\item Se tiene una imagen del plano o un pdf del mismo.
	\end{itemize}
	\\
	\textbf{Acciones} 				& 
	\begin{enumerate}
		\item El usuario accede a la herramienta Architect.
		\item El usuario se autentica.
		\item El usuario pincha sobre el icono de edificio que aparece en el plano de localización, si no, debe emplazar un edificio.
		\item El usuario pincha en la pestaña ``Floors" de su panel de control.
		\item El usuario pincha en el text-area de ``Floor Number'', para introducir a que planta pertenece el plano.
		\item El usuario pincha en el botón de ``Floor Plan''.
		\item El usuario elige el plano a introducir.
		\item El usuario pulsa sobre el botón \checkmark para emplazar definitivamente el plano.
	\end{enumerate}
	\\
	
	\textbf{Postcondiciones} 		& 
	\begin{itemize}
		\item El plano del edificio se almacena en la base de datos.
		\item La información posteriormente será visible.
	\end{itemize}
	\\
	\textbf{Excepciones} 			& 
	\begin{itemize}
		\item Las bases de datos no están disponibles.
		\item Las REST API no están disponibles.
		\item La información introducida no es válida
	\end{itemize}
	
	\\
	\textbf{Importancia} 			& Alta\\}

%%%%%%%%%%%%%%%%%%%%%%%%%%%%%%%%%%%%%%%%%%%%%%%%%%%%%%%%%%%%%%%%%%%%%%%%%%%%%%%%%%%%%%%%%%%%%%%%%%%%%%%%%%%%%%%%%%%%%%%%%%%%%

\tablaSinColores{CU-05 Redimensionar y mover plano}
{L{3.5cm} L{10cm}}
{2}
{Tabla CU-05}
{\textbf{CU-05} & \textbf{Redimensionar y mover plano} \\}
{\textbf{Versión} 				& 1.0\\ 
	\textbf{Autor} 				& Juan Pedro Pascual Vitores\\
	\textbf{Requisitos asociados} 	& RF-1.4\\
	\textbf{Descripción} 			& 
	Permite al usuario el movimiento y redimensión del plano añadido.\\
	\textbf{Precondiciones} 		& 
	\begin{itemize}
		\item Se encuentran disponibles las bases de datos.
		\item Se encuentran disponibles las REST API.
		\item Se accede como usuario Architect (autenticado).
		\item Se tiene una imagen del plano o un pdf del mismo.
	\end{itemize}
	\\
	\textbf{Acciones} 				& 
	\begin{enumerate}
		\item El usuario accede a la herramienta Architect.
		\item El usuario se autentica.
		\item El usuario pincha sobre el icono de edificio que aparece en el plano de localización, si no, debe emplazar un edificio.
		\item El usuario pincha en la pestaña ``Floors" de su panel de control.
		\item El usuario pincha en el text-area de ``Floor Number'', para introducir a que planta pertenece el plano.
		\item El usuario pincha en el botón de ``Floor Plan''.
		\item El usuario elige el plano a introducir.
		\item El usuario mueve y redimensiona la imagen a través de los iconos típicos de redimensión y giro de imagen.
		\item El usuario pulsa sobre el botón \checkmark para emplazar definitivamente el plano.
	\end{enumerate}
	\\
	
	\textbf{Postcondiciones} 		& 
	\begin{itemize}
		\item El plano del edificio se almacena en la base de datos.
		\item La información posteriormente será visible.
	\end{itemize}
	\\
	\textbf{Excepciones} 			& 
	\begin{itemize}
		\item Las bases de datos no están disponible.
		\item Las REST API no están disponibles.
		\item La información introducida no es válida
	\end{itemize}
	
	\\
	\textbf{Importancia} 			& Alta\\}

%%%%%%%%%%%%%%%%%%%%%%%%%%%%%%%%%%%%%%%%%%%%%%%%%%%%%%%%%%%%%%%%%%%%%%%%%%%%%%%%%%%%%%%%%%%%%%%%%%%%%%%%%%%%%%%%%%%%%%%%%%%%%

\tablaSinColores{CU-06 Borrar edificio}
{L{3.5cm} L{10cm}}
{2}
{Tabla CU-06}
{\textbf{CU-06} & \textbf{Borrar edificio} \\}
{\textbf{Versión} 				& 1.0\\ 
	\textbf{Autor} 				& Juan Pedro Pascual Vitores\\
	\textbf{Requisitos asociados} 	& RF-1.5\\
	\textbf{Descripción} 			& 
	Permite al usuario el borrado de un edificio.\\
	\textbf{Precondiciones} 		& 
	\begin{itemize}
		\item Se encuentran disponibles las bases de datos.
		\item Se encuentran disponibles las REST APIs.
		\item Se accede como usuario Architect (autenticado).
	\end{itemize}
	\\
	\textbf{Acciones} 				& 
	\begin{enumerate}
		\item El usuario accede a la herramienta Architect.
		\item El usuario se autentica.
		\item El usuario pincha sobre el icono de edificio que aparece en el plano de localización.
		\item El usuario pincha en la pestaña ``Buildings" de su panel de control.
		\item El usuario pincha en el botón ``Delete'', y posteriormente en el botón ``Confirm Deletion" para borrar de forma definitiva el edificio.
	\end{enumerate}
	\\
	
	\textbf{Postcondiciones} 		& 
	\begin{itemize}
		\item La información del edificio posteriormente no será visible.
	\end{itemize}
	\\
	\textbf{Excepciones} 			& 
	\begin{itemize}
		\item Las bases de datos no están disponibles.
		\item Las REST API no están disponibles.
	\end{itemize}
	
	\\
	\textbf{Importancia} 			& Alta\\}

%%%%%%%%%%%%%%%%%%%%%%%%%%%%%%%%%%%%%%%%%%%%%%%%%%%%%%%%%%%%%%%%%%%%%%%%%%%%%%%%%%%%%%%%%%%%%%%%%%%%%%%%%%%%%%%%%%%%%%%%%%%%%

\tablaSinColores{CU-07 Gestión de pois}
{L{3.5cm} L{10cm}}
{2}
{Tabla CU-07}
{\textbf{CU-07} & \textbf{Gestión de pois} \\}
{\textbf{Versión} 				& 1.0\\ 
	\textbf{Autor} 				& Juan Pedro Pascual Vitores\\
	\textbf{Requisitos asociados} 	& RF-2, RF-2.1, RF-2.2, RF-2.3, RF-2.4\\
	\textbf{Descripción} 			& 
	Permite al usuario la gestión y administración de pois.\\
	\textbf{Precondiciones} 		& 
	\begin{itemize}
		\item Se encuentran disponibles las bases de datos.
		\item Se encuentran disponibles las REST API.
		\item Se accede como usuario Architect (autenticado).
	\end{itemize}
	\\
	\textbf{Acciones} 				& 
	\begin{enumerate}
		\item El usuario accede a la herramienta Architect.
		\item El usuario se autentica.
		\item El usuario pincha sobre el icono de edificio que aparece en el plano de localización, si no, debe emplazar un edificio y su respectivo plano.
		\item El usuario pincha en la pestaña ``POIs" de su panel de control.
		\item El usuario interacciona con la herramienta para introducir la información concerniente a los pois de un edificio.
	\end{enumerate}
	\\
	
	\textbf{Postcondiciones} 		& 
	\begin{itemize}
		\item La información sobre los pois del edificio se almacena en la base de datos.
		\item La información de los pois del edificio será visible.
	\end{itemize}
	\\
	\textbf{Excepciones} 			& 
	\begin{itemize}
		\item Las bases de datos no están disponibles.
		\item Las REST API no está disponibles.
		\item No hay edificio donde ubicar los pois.
	\end{itemize}
	
	\\
	\textbf{Importancia} 			& Alta\\}

%%%%%%%%%%%%%%%%%%%%%%%%%%%%%%%%%%%%%%%%%%%%%%%%%%%%%%%%%%%%%%%%%%%%%%%%%%%%%%%%%%%%%%%%%%%%%%%%%%%%%%%%%%%%%%%%%%%%%%%%%%%%%

\tablaSinColores{CU-08 Emplazamiento de pois}
{L{3.5cm} L{10cm}}
{2}
{Tabla CU-08}
{\textbf{CU-08} & \textbf{Emplazamiento de pois} \\}
{\textbf{Versión} 				& 1.0\\ 
	\textbf{Autor} 				& Juan Pedro Pascual Vitores\\
	\textbf{Requisitos asociados} 	& RF-2.1\\
	\textbf{Descripción} 			& 
	Permite al usuario la colocación de puntos de interés sobre el plano.\\
	\textbf{Precondiciones} 		& 
	\begin{itemize}
		\item Se encuentran disponibles las bases de datos.
		\item Se encuentran disponibles las REST API.
		\item Se accede como usuario Architect (autenticado).
	\end{itemize}
	\\
	\textbf{Acciones} 				& 
	\begin{enumerate}
		\item El usuario accede a la herramienta Architect.
		\item El usuario se autentica.
		\item El usuario pincha sobre el icono de edificio que aparece en el plano de localización, si no, debe emplazar un edificio y su respectivo plano.
		\item El usuario pincha en la pestaña ``POIs" de su panel de control.
		\item El usuario pincha en el icono de ``drag to add new POI'' y arrastra hasta una posición del plano del edificio.
	\end{enumerate}
	\\
	
	\textbf{Postcondiciones} 		& 
	\begin{itemize}
		\item Rellenar por parte del usuario, la información imprescindible del poi para su validación.
		\item La información sobre los pois del edificio se almacena en la base de datos.
		\item La información de los pois del edificio será visible.
	\end{itemize}
	\\
	\textbf{Excepciones} 			& 
	\begin{itemize}
		\item Las bases de datos no están disponibles.
		\item Las REST API no está disponibles.
		\item No hay edificio donde ubicar los pois.
	\end{itemize}
	
	\\
	\textbf{Importancia} 			& Alta\\}

%%%%%%%%%%%%%%%%%%%%%%%%%%%%%%%%%%%%%%%%%%%%%%%%%%%%%%%%%%%%%%%%%%%%%%%%%%%%%%%%%%%%%%%%%%%%%%%%%%%%%%%%%%%%%%%%%%%%%%%%%%%%%

\tablaSinColores{CU-09 Detallar información de pois}
{L{3.5cm} L{10cm}}
{2}
{Tabla CU-09}
{\textbf{CU-09} & \textbf{Detallar información de pois} \\}
{\textbf{Versión} 				& 1.0\\ 
	\textbf{Autor} 				& Juan Pedro Pascual Vitores\\
	\textbf{Requisitos asociados} 	& RF-2.2\\
	\textbf{Descripción} 			& 
	Permite al usuario adjuntar información al poi (nombre, descripción, tipo, etc\ldots).\\
	\textbf{Precondiciones} 		& 
	\begin{itemize}
		\item Se encuentran disponibles las bases de datos.
		\item Se encuentran disponibles las REST API.
		\item Se accede como usuario Architect (autenticado).
	\end{itemize}
	\\
	\textbf{Acciones} 				& 
	\begin{enumerate}
		\item El usuario accede a la herramienta Architect.
		\item El usuario se autentica.
		\item El usuario pincha sobre el icono de edificio que aparece en el plano de localización, si no, debe emplazar un edificio y su respectivo plano.
		\item El usuario pincha en la pestaña ``POIs" de su panel de control.
		\item El usuario pincha en el icono de ``drag to add new POI'' y arrastra hasta una posición del plano del edificio.
		\item El usuario en el panel emergente tiene varios campos que puede rellenar para adjuntar información al poi.
		\item El usuario pincha en el botón ``Add" para añadir de forma definitiva el poi al plano y al edificio .
	\end{enumerate}
	\\
	
	\textbf{Postcondiciones} 		& 
	\begin{itemize}
		\item La información sobre los pois del edificio se almacena en la base de datos.
		\item La información de los pois del edificio será visible.
	\end{itemize}
	\\
	\textbf{Excepciones} 			& 
	\begin{itemize}
		\item Las bases de datos no están disponibles.
		\item Las REST API no está disponibles.
		\item No hay edificio donde ubicar los pois.
	\end{itemize}
	
	\\
	\textbf{Importancia} 			& Alta\\}

%%%%%%%%%%%%%%%%%%%%%%%%%%%%%%%%%%%%%%%%%%%%%%%%%%%%%%%%%%%%%%%%%%%%%%%%%%%%%%%%%%%%%%%%%%%%%%%%%%%%%%%%%%%%%%%%%%%%%%%%%%%%%

\tablaSinColores{CU-10 Ubicar pois}
{L{3.5cm} L{10cm}}
{2}
{Tabla CU-10}
{\textbf{CU-10} & \textbf{Ubicar pois} \\}
{\textbf{Versión} 				& 1.0\\ 
	\textbf{Autor} 				& Juan Pedro Pascual Vitores\\
	\textbf{Requisitos asociados} 	& RF-2.3\\
	\textbf{Descripción} 			& 
	Permite al usuario la colocación del poi sobre una localización del plano de un edifico.\\
	\textbf{Precondiciones} 		& 
	\begin{itemize}
		\item Se encuentran disponibles las bases de datos.
		\item Se encuentran disponibles las REST API.
		\item Se accede como usuario Architect (autenticado).
	\end{itemize}
	\\
	\textbf{Acciones} 				& 
	\begin{enumerate}
		\item El usuario accede a la herramienta Architect.
		\item El usuario se autentica.
		\item El usuario pincha sobre el icono de edificio que aparece en el plano de localización, si no, debe emplazar un edificio y su respectivo plano.
		\item El usuario pincha en la pestaña ``POIs" de su panel de control.
		\item El usuario pincha en el icono de ``drag to add new POI'' y arrastra hasta una posición del plano del edificio.
		\item El usuario pincha en el poi y sin soltar lo cambia de localización, dejando de mantener el click para dejar el poi en la posición deseada.
	\end{enumerate}
	\\
	
	\textbf{Postcondiciones} 		& 
	\begin{itemize}
		\item La información sobre los pois del edificio se almacena en la base de datos.
		\item La información de los pois del edificio será visible.
	\end{itemize}
	\\
	\textbf{Excepciones} 			& 
	\begin{itemize}
		\item Las bases de datos no están disponibles.
		\item Las REST API no está disponibles.
		\item No hay edificio donde ubicar los pois.
	\end{itemize}
	
	\\
	\textbf{Importancia} 			& Alta\\}

%%%%%%%%%%%%%%%%%%%%%%%%%%%%%%%%%%%%%%%%%%%%%%%%%%%%%%%%%%%%%%%%%%%%%%%%%%%%%%%%%%%%%%%%%%%%%%%%%%%%%%%%%%%%%%%%%%%%%%%%%%%%%

\tablaSinColores{CU-11 Enlazar pois}
{L{3.5cm} L{10cm}}
{2}
{Tabla CU-11}
{\textbf{CU-11} & \textbf{Enlazar pois} \\}
{\textbf{Versión} 				& 1.0\\ 
	\textbf{Autor} 				& Juan Pedro Pascual Vitores\\
	\textbf{Requisitos asociados} 	& RF-2.4\\
	\textbf{Descripción} 			& 
	Permite al usuario crear un camino entre pois.\\
	\textbf{Precondiciones} 		& 
	\begin{itemize}
		\item Se encuentran disponibles las bases de datos.
		\item Se encuentran disponibles las REST API.
		\item Se accede como usuario Architect (autenticado).
		\item Hay dos o más pois creados en un plano del edificio.
	\end{itemize}
	\\
	\textbf{Acciones} 				& 
	\begin{enumerate}
		\item El usuario accede a la herramienta Architect.
		\item El usuario se autentica.
		\item El usuario pincha sobre el icono de edificio que aparece en el plano de localización, si no, debe emplazar un edificio y su respectivo plano.
		\item El usuario pincha en la pestaña ``POIs" de su panel de control.
		\item El usuario pincha sobre el icono de ``toggle edge mode'' para poner el modo camino a ON
		\item El usuario pincha sobre un poi, y después pincha sobre otro para crear un camino en linea recta entre ellos. (Si se desea hacer caminos entre pois con curvas, existen pois intermedios llamados conectores que lo permiten.)
	\end{enumerate}
	\\
	
	\textbf{Postcondiciones} 		& 
	\begin{itemize}
		\item La información sobre los pois del edificio se almacena en la base de datos.
		\item La información de los pois del edificio será visible.
	\end{itemize}
	\\
	\textbf{Excepciones} 			& 
	\begin{itemize}
		\item Las bases de datos no están disponibles.
		\item Las REST API no está disponibles.
		\item No hay edificio donde ubicar los pois.
	\end{itemize}
	
	\\
	\textbf{Importancia} 			& Alta\\}

%%%%%%%%%%%%%%%%%%%%%%%%%%%%%%%%%%%%%%%%%%%%%%%%%%%%%%%%%%%%%%%%%%%%%%%%%%%%%%%%%%%%%%%%%%%%%%%%%%%%%%%%%%%%%%%%%%%%%%%%%%%%%

\tablaSinColores{CU-12 Gestión de rutas}
{L{3.5cm} L{10cm}}
{2}
{Tabla CU-12}
{\textbf{CU-12} & \textbf{Gestión de rutas} \\}
{\textbf{Versión} 				& 1.0\\ 
	\textbf{Autor} 				& Juan Pedro Pascual Vitores\\
	\textbf{Requisitos asociados} 	& RF-3, RF-3.1, RF-3.2, RF-3.3, RF-3.4, RF-3.5, RF-3.6, RF-3.7\\
	\textbf{Descripción} 			& 
	Permite al usuario la gestión y administración de rutas predefinidas.\\
	\textbf{Precondiciones} 		& 
	\begin{itemize}
		\item Se encuentran disponibles las bases de datos.
		\item Se encuentran disponibles las REST API.
		\item Se accede como usuario Architect (autenticado).
		\item Hay dos o más pois creados en un plano del edificio.
	\end{itemize}
	\\
	\textbf{Acciones} 				& 
	\begin{enumerate}
		\item El usuario accede a la herramienta Architect.
		\item El usuario se autentica.
		\item El usuario pincha sobre el icono de edificio que aparece en el plano de localización, si no, debe emplazar un edificio y su respectivo plano.
		\item El usuario pincha en la pestaña ``POIs" de su panel de control.
		\item El usuario interacciona con la herramienta para realizar la gestión de las rutas.
	\end{enumerate}
	\\
	
	\textbf{Postcondiciones} 		& 
	\begin{itemize}
		\item La información sobre las rutas del edificio se almacena en la base de datos.
		\item La información de las rutas del edificio será visible.
	\end{itemize}
	\\
	\textbf{Excepciones} 			& 
	\begin{itemize}
		\item Las bases de datos no están disponibles.
		\item Las REST API no está disponibles.
		\item No hay edificio donde ubicar las rutas y pois para formarlas.
	\end{itemize}
	
	\\
	\textbf{Importancia} 			& Alta\\}

%%%%%%%%%%%%%%%%%%%%%%%%%%%%%%%%%%%%%%%%%%%%%%%%%%%%%%%%%%%%%%%%%%%%%%%%%%%%%%%%%%%%%%%%%%%%%%%%%%%%%%%%%%%%%%%%%%%%%%%%%%%%%

\tablaSinColores{CU-13 Crear ruta}
{L{3.5cm} L{10cm}}
{2}
{Tabla CU-13}
{\textbf{CU-13} & \textbf{Crear ruta} \\}
{\textbf{Versión} 				& 1.0\\ 
	\textbf{Autor} 				& Juan Pedro Pascual Vitores\\
	\textbf{Requisitos asociados} 	& RF-3.1\\
	\textbf{Descripción} 			& 
	Permite al usuario la creación de rutas predefinidas.\\
	\textbf{Precondiciones} 		& 
	\begin{itemize}
		\item Se encuentran disponibles las bases de datos.
		\item Se encuentran disponibles las REST API.
		\item Se accede como usuario Architect (autenticado).
		\item Hay dos o más pois creados en un plano del edificio.
	\end{itemize}
	\\
	\textbf{Acciones} 				& 
	Se especifican todos los pasos necesarios para crear una ruta, teniendo en cuenta restricciones como; que el nombre de la ruta no este vacío, que una ruta tenga más de un poi, que no se repitan pois en la ruta etc\ldots
	\begin{enumerate}
		\item El usuario accede a la herramienta Architect.
		\item El usuario se autentica.
		\item El usuario pincha sobre el icono de edificio que aparece en el plano de localización, si no, debe emplazar un edificio y su respectivo plano.
		\item El usuario pincha en la pestaña ``POIs" de su panel de control.
		\item El usuario pincha en el icono de  ``add route mode".
		\item El usuario rellena el campo que acaba de aparecer con el nombre de la ruta que desea crear.
		\item El usuario pincha en los pois que quiere añadir a la ruta, y en el panel emergente, pulsa el botón ``add POI to the route" para añadirlos.
		\item El usuario pulsa el botón ``Create route" para crear una ruta predefinida de forma definitiva.
	\end{enumerate}
	\\
	
	\textbf{Postcondiciones} 		& 
	\begin{itemize}
		\item La información sobre las rutas del edificio se almacena en la base de datos.
		\item La información de las rutas del edificio será visible.
	\end{itemize}
	\\
	\textbf{Excepciones} 			& 
	\begin{itemize}
		\item Las bases de datos no están disponibles.
		\item Las REST API no está disponibles.
		\item No hay edificio donde ubicar las rutas y pois para formarlas.
	\end{itemize}
	
	\\
	\textbf{Importancia} 			& Alta\\}

%%%%%%%%%%%%%%%%%%%%%%%%%%%%%%%%%%%%%%%%%%%%%%%%%%%%%%%%%%%%%%%%%%%%%%%%%%%%%%%%%%%%%%%%%%%%%%%%%%%%%%%%%%%%%%%%%%%%%%%%%%%%%

\tablaSinColores{CU-14 Detallar información de la ruta}
{L{3.5cm} L{10cm}}
{2}
{Tabla CU-14}
{\textbf{CU-14} & \textbf{Detallar información de la ruta} \\}
{\textbf{Versión} 				& 1.0\\ 
	\textbf{Autor} 				& Juan Pedro Pascual Vitores\\
	\textbf{Requisitos asociados} 	& RF-3.2\\
	\textbf{Descripción} 			& 
	Permite al usuario agregar información a la ruta, como el nombre o la planta.\\
	\textbf{Precondiciones} 		& 
	\begin{itemize}
		\item Se encuentran disponibles las bases de datos.
		\item Se encuentran disponibles las REST API.
		\item Se accede como usuario Architect (autenticado).
		\item Hay dos o más pois creados en un plano del edificio.
	\end{itemize}
	\\
	\textbf{Acciones} 				& 
	\begin{enumerate}
		\item El usuario accede a la herramienta Architect.
		\item El usuario se autentica.
		\item El usuario pincha sobre el icono de edificio que aparece en el plano de localización, si no, debe emplazar un edificio y su respectivo plano.
		\item El usuario pincha en la pestaña ``POIs" de su panel de control.
		\item El usuario pincha en el icono de  ``add route mode".
		\item El usuario rellena el campo que acaba de aparecer con el nombre de la ruta que desea crear.
		\item El usuario pincha en los pois que quiere añadir a la ruta, y en el panel emergente, pulsa el botón ``add POI to the route" para añadirlos.
		\item El usuario pulsa el botón ``Create route" para crear una ruta predefinida de forma definitiva. En este momento es en el que se le agrega de forma automática el número de planta.
	\end{enumerate}
	\\
	
	\textbf{Postcondiciones} 		& 
	\begin{itemize}
		\item La información sobre las rutas del edificio se almacena en la base de datos.
		\item La información de las rutas del edificio será visible.
	\end{itemize}
	\\
	\textbf{Excepciones} 			& 
	\begin{itemize}
		\item Las bases de datos no están disponibles.
		\item Las REST API no está disponibles.
		\item No hay edificio donde ubicar las rutas y pois para formarlas.
	\end{itemize}
	
	\\
	\textbf{Importancia} 			& Alta\\}

%%%%%%%%%%%%%%%%%%%%%%%%%%%%%%%%%%%%%%%%%%%%%%%%%%%%%%%%%%%%%%%%%%%%%%%%%%%%%%%%%%%%%%%%%%%%%%%%%%%%%%%%%%%%%%%%%%%%%%%%%%%%%

\tablaSinColores{CU-15 Añadir pois a la ruta}
{L{3.5cm} L{10cm}}
{2}
{Tabla CU-15}
{\textbf{CU-15} & \textbf{Añadir pois a la ruta} \\}
{\textbf{Versión} 				& 1.0\\ 
	\textbf{Autor} 				& Juan Pedro Pascual Vitores\\
	\textbf{Requisitos asociados} 	& RF-3.3\\
	\textbf{Descripción} 			& 
	Permite al usuario añadir puntos de interés a la ruta.\\
	\textbf{Precondiciones} 		& 
	\begin{itemize}
		\item Se encuentran disponibles las bases de datos.
		\item Se encuentran disponibles las REST API.
		\item Se accede como usuario Architect (autenticado).
		\item Hay dos o más pois creados en un plano del edificio.
	\end{itemize}
	\\
	\textbf{Acciones} 				& 
	\begin{enumerate}
		\item El usuario accede a la herramienta Architect.
		\item El usuario se autentica.
		\item El usuario pincha sobre el icono de edificio que aparece en el plano de localización, si no, debe emplazar un edificio y su respectivo plano.
		\item El usuario pincha en la pestaña ``POIs" de su panel de control.
		\item El usuario pincha en el icono de  ``add route mode".
		\item El usuario pincha en los pois que quiere añadir a la ruta, y en el panel emergente, pulsa el botón ``add POI to the route" para añadirlos.
	\end{enumerate}
	\\
	
	\textbf{Postcondiciones} 		& 
	\begin{itemize}
		\item La información sobre las rutas del edificio se almacena en la base de datos.
		\item La información de las rutas del edificio será visible.
	\end{itemize}
	\\
	\textbf{Excepciones} 			& 
	\begin{itemize}
		\item Las bases de datos no están disponibles.
		\item Las REST API no está disponibles.
		\item No hay edificio donde ubicar las rutas y pois para formarlas.
	\end{itemize}
	
	\\
	\textbf{Importancia} 			& Alta\\}

%%%%%%%%%%%%%%%%%%%%%%%%%%%%%%%%%%%%%%%%%%%%%%%%%%%%%%%%%%%%%%%%%%%%%%%%%%%%%%%%%%%%%%%%%%%%%%%%%%%%%%%%%%%%%%%%%%%%%%%%%%%%%

\tablaSinColores{CU-16 Trazar ruta}
{L{3.5cm} L{10cm}}
{2}
{Tabla CU-16}
{\textbf{CU-16} & \textbf{Trazar ruta} \\}
{\textbf{Versión} 				& 1.0\\ 
	\textbf{Autor} 				& Juan Pedro Pascual Vitores\\
	\textbf{Requisitos asociados} 	& RF-3.4\\
	\textbf{Descripción} 			& 
	Permite al usuario ver el trazado de la ruta que está creando.\\
	\textbf{Precondiciones} 		& 
	\begin{itemize}
		\item Se encuentran disponibles las bases de datos.
		\item Se encuentran disponibles las REST API.
		\item Se accede como usuario Architect (autenticado).
		\item Hay dos o más pois creados en un plano del edificio.
	\end{itemize}
	\\
	\textbf{Acciones} 				& 
	\begin{enumerate}
		\item El usuario accede a la herramienta Architect.
		\item El usuario se autentica.
		\item El usuario pincha sobre el icono de edificio que aparece en el plano de localización, si no, debe emplazar un edificio y su respectivo plano.
		\item El usuario pincha en la pestaña ``POIs" de su panel de control.
		\item El usuario pincha en el icono de ``add route mode".
		\item El usuario pincha en los pois que quiere añadir a la ruta, y en el panel emergente, pulsa el botón ``add POI to the route" para añadirlos. Cada vez que el usuario añada un poi se irá mostrando el trazado de la ruta.
	\end{enumerate}
	\\
	
	\textbf{Postcondiciones} 		& 
	\begin{itemize}
		\item La información sobre las rutas del edificio se almacena en la base de datos.
		\item La información de las rutas del edificio será visible.
	\end{itemize}
	\\
	\textbf{Excepciones} 			& 
	\begin{itemize}
		\item Las bases de datos no están disponibles.
		\item Las REST API no está disponibles.
		\item No hay edificio donde ubicar las rutas y pois para formarlas.
	\end{itemize}
	
	\\
	\textbf{Importancia} 			& Alta\\}

%%%%%%%%%%%%%%%%%%%%%%%%%%%%%%%%%%%%%%%%%%%%%%%%%%%%%%%%%%%%%%%%%%%%%%%%%%%%%%%%%%%%%%%%%%%%%%%%%%%%%%%%%%%%%%%%%%%%%%%%%%%%%

\tablaSinColores{CU-17 Seleccionar ruta}
{L{3.5cm} L{10cm}}
{2}
{Tabla CU-17}
{\textbf{CU-17} & \textbf{Seleccionar ruta} \\}
{\textbf{Versión} 				& 1.0\\ 
	\textbf{Autor} 				& Juan Pedro Pascual Vitores\\
	\textbf{Requisitos asociados} 	& RF-3.5\\
	\textbf{Descripción} 			& 
	Permite al usuario elegir entre las diferentes rutas existentes, ya sea para mostrar su trazado o modificarlo.\\
	\textbf{Precondiciones} 		& 
	\begin{itemize}
		\item Se encuentran disponibles las bases de datos.
		\item Se encuentran disponibles las REST API.
		\item Se accede como usuario Architect (autenticado).
		\item Hay rutas creadas en el edificio.
	\end{itemize}
	\\
	\textbf{Acciones} 				& 
	\begin{enumerate}
		\item El usuario accede a la herramienta Architect.
		\item El usuario se autentica.
		\item El usuario pincha sobre el icono de edificio que aparece en el plano de localización, si no, debe emplazar un edificio, su respectivo plano y crear al menos una ruta predefinida.
		\item El usuario pincha en el desplegable que se encuentra arriba a la izquierda, y selecciona una ruta, se mostrará su trazado. Para editarla se debe ir a la pestaña ``POIs" del panel de control, activar el botón ``add route mode" y ya se permitiría su edición.
	\end{enumerate}
	\\
	
	\textbf{Postcondiciones} 		& 
	\begin{itemize}
		\item La información sobre las rutas del edificio se almacena en la base de datos, si se edita la ruta.
		\item La información de las rutas del edificio será visible.
	\end{itemize}
	\\
	\textbf{Excepciones} 			& 
	\begin{itemize}
		\item Las bases de datos no están disponibles.
		\item Las REST API no está disponibles.
	\end{itemize}
	
	\\
	\textbf{Importancia} 			& Alta\\}

%%%%%%%%%%%%%%%%%%%%%%%%%%%%%%%%%%%%%%%%%%%%%%%%%%%%%%%%%%%%%%%%%%%%%%%%%%%%%%%%%%%%%%%%%%%%%%%%%%%%%%%%%%%%%%%%%%%%%%%%%%%%%

\tablaSinColores{CU-18 Limpiar ruta}
{L{3.5cm} L{10cm}}
{2}
{Tabla CU-18}
{\textbf{CU-18} & \textbf{Limpiar ruta} \\}
{\textbf{Versión} 				& 1.0\\ 
	\textbf{Autor} 				& Juan Pedro Pascual Vitores\\
	\textbf{Requisitos asociados} 	& RF-3.6\\
	\textbf{Descripción} 			& 
	Permite al usuario dejar de ver el trazado de la ruta.\\
	\textbf{Precondiciones} 		& 
	\begin{itemize}
		\item Se encuentran disponibles las bases de datos.
		\item Se encuentran disponibles las REST API.
		\item Se accede como usuario Architect (autenticado).
		\item Hay rutas creadas en el edificio.
	\end{itemize}
	\\
	\textbf{Acciones} 				& 
	\begin{enumerate}
		\item El usuario accede a la herramienta Architect.
		\item El usuario se autentica.
		\item El usuario pincha sobre el icono de edificio que aparece en el plano de localización, si no, debe emplazar un edificio, su respectivo plano y crear al menos una ruta predefinida.
		\item El usuario pincha en el desplegable que se encuentra arriba a la izquierda, y selecciona una ruta, se mostrará su trazado.
		\item El usuario pincha en el botón ``Clear" que se encuentra arriba a la izquierda, para dejar de ver el trazado de la ruta seleccionada.
	\end{enumerate}
	\\
	
	\textbf{Postcondiciones} 		& 
	\begin{itemize}
		\item No existe mayor repercusión más allá de la visual.
	\end{itemize}
	\\
	\textbf{Excepciones} 			& 
	\begin{itemize}
		\item Las bases de datos no están disponibles.
		\item Las REST API no está disponibles.
	\end{itemize}
	
	\\
	\textbf{Importancia} 			& Alta\\}

%%%%%%%%%%%%%%%%%%%%%%%%%%%%%%%%%%%%%%%%%%%%%%%%%%%%%%%%%%%%%%%%%%%%%%%%%%%%%%%%%%%%%%%%%%%%%%%%%%%%%%%%%%%%%%%%%%%%%%%%%%%%%

\tablaSinColores{CU-19 Borrar ruta}
{L{3.5cm} L{10cm}}
{2}
{Tabla CU-19}
{\textbf{CU-19} & \textbf{Borrar ruta} \\}
{\textbf{Versión} 				& 1.0\\ 
	\textbf{Autor} 				& Juan Pedro Pascual Vitores\\
	\textbf{Requisitos asociados} 	& RF-3.7\\
	\textbf{Descripción} 			& 
	Permite al usuario el borrado de una ruta.\\
	\textbf{Precondiciones} 		& 
	\begin{itemize}
		\item Se encuentran disponibles las bases de datos.
		\item Se encuentran disponibles las REST API.
		\item Se accede como usuario Architect (autenticado).
		\item Hay rutas creadas en el edificio.
	\end{itemize}
	\\
	\textbf{Acciones} 				& 
	\begin{enumerate}
		\item El usuario accede a la herramienta Architect.
		\item El usuario se autentica.
		\item El usuario pincha sobre el icono de edificio que aparece en el plano de localización, si no, debe emplazar un edificio, su respectivo plano y crear al menos una ruta predefinida.
		\item El usuario pincha en el desplegable que se encuentra arriba a la izquierda, y selecciona una ruta, se mostrará su trazado.
		\item El usuario pincha en el botón ``x" que se encuentra arriba a la izquierda, para borrar de forma definitiva la ruta seleccionada.
	\end{enumerate}
	\\
	
	\textbf{Postcondiciones} 		& 
	\begin{itemize}
		\item La información de la ruta borrada no será visible.
	\end{itemize}
	\\
	\textbf{Excepciones} 			& 
	\begin{itemize}
		\item Las bases de datos no están disponibles.
		\item Las REST API no está disponibles.
	\end{itemize}
	
	\\
	\textbf{Importancia} 			& Alta\\}

%%%%%%%%%%%%%%%%%%%%%%%%%%%%%%%%%%%%%%%%%%%%%%%%%%%%%%%%%%%%%%%%%%%%%%%%%%%%%%%%%%%%%%%%%%%%%%%%%%%%%%%%%%%%%%%%%%%%%%%%%%%%%
%%%%%%%%%%%%%%%%%%%%%%%%%%%%%%%%%%%%%%%%%%%%%%%%%%%%%%%%%%%%%%%%%%%%%%%%%%%%%%%%%%%%%%%%%%%%%%%%%%%%%%%%%%%%%%%%%%%%%%%%%%%%%
%%%%%%%%%%%%%%%%%%%%%%%%%%%%%%%%%%%%%%%%%%%%%%%%%%%%%%%%%%%%%%%%%%%%%%%%%%%%%%%%%%%%%%%%%%%%%%%%%%%%%%%%%%%%%%%%%%%%%%%%%%%%%

\tablaSinColores{CU-20 Visualización de edificios}
{L{3.5cm} L{10cm}}
{2}
{Tabla CU-20}
{\textbf{CU-20} & \textbf{Visualización de edificios} \\}
{\textbf{Versión} 				& 1.0\\ 
	\textbf{Autor} 				& Juan Pedro Pascual Vitores\\
	\textbf{Requisitos asociados} 	& RF-4, RF-4.1, RF-4.2\\
	\textbf{Descripción} 			& 
	Permite al usuario el borrado de una ruta.\\
	\textbf{Precondiciones} 		& 
	\begin{itemize}
		\item Se encuentran disponibles las bases de datos.
		\item Se encuentran disponibles las REST API.
		\item Se accede como usuario Viewer (no autenticado).
		\item Hay edificios creados.
	\end{itemize}
	\\
	\textbf{Acciones} 				& 
	El usuario puede localizar un edificio a través de un desplegable que se encuentra en la parte superior de la interfaz, además de la manera que se va explicar a continuación. 
	\begin{enumerate}
		\item El usuario accede a la herramienta Viewer.
		\item El usuario pincha sobre clusters que aparecen en el plano de localización hasta ver un icono ``G'', el cual significa que, ahí se encuentra un edificio.
		\item El usuario pincha sobre el icono ``G'' y podrá visualizar la información sobre el edificio.
	\end{enumerate}
	\\
	
	\textbf{Postcondiciones} 		& 
	\begin{itemize}
		\item La información del edificio se ve tal como el usuario administrador (Architect), la ha introducido.
	\end{itemize}
	\\
	\textbf{Excepciones} 			& 
	\begin{itemize}
		\item Las bases de datos no están disponibles.
		\item Las REST API no está 
		\item No existen edificios.
	\end{itemize}
	
	\\
	\textbf{Importancia} 			& Alta\\}

%%%%%%%%%%%%%%%%%%%%%%%%%%%%%%%%%%%%%%%%%%%%%%%%%%%%%%%%%%%%%%%%%%%%%%%%%%%%%%%%%%%%%%%%%%%%%%%%%%%%%%%%%%%%%%%%%%%%%%%%%%%%%

\tablaSinColores{CU-21 Buscar edificio}
{L{3.5cm} L{10cm}}
{2}
{Tabla CU-21}
{\textbf{CU-21} & \textbf{Buscar edificio} \\}
{\textbf{Versión} 				& 1.0\\ 
	\textbf{Autor} 				& Juan Pedro Pascual Vitores\\
	\textbf{Requisitos asociados} 	& RF-4.1\\
	\textbf{Descripción} 			& 
	Permite al usuario encontrar y visualizar un edificio a partir de su identificador o nombre.\\
	\textbf{Precondiciones} 		& 
	\begin{itemize}
		\item Se encuentran disponibles las bases de datos.
		\item Se encuentran disponibles las REST API.
		\item Se accede como usuario Viewer (no autenticado).
		\item Hay edificios creados.
	\end{itemize}
	\\
	\textbf{Acciones} 				&  
	\begin{enumerate}
		\item El usuario accede a la herramienta Viewer.
		\item El usuario tiene 2 opciones para buscar un edificio:
		\begin{itemize}
			\item Localizando el edificio en el mapa: el usuario pincha sobre clusters que aparecen en el plano de localización hasta ver un icono ``G", el cual significa que, ahí se encuentra un edificio.
			\item Mediante el nombre o el identificador del edificio: el usuario puede localizar un edificio a través de un desplegable que se encuentra en la parte superior de la interfaz, introduciendo su nombre o identificador.
		\end{itemize}
	\end{enumerate}
	\\
	
	\textbf{Postcondiciones} 		& 
	\begin{itemize}
		\item La información del edificio se ve tal como el usuario administrador (Architect), la ha introducido.
	\end{itemize}
	\\
	\textbf{Excepciones} 			& 
	\begin{itemize}
		\item Las bases de datos no están disponibles.
		\item Las REST API no está 
		\item No existen edificios.
	\end{itemize}
	
	\\
	\textbf{Importancia} 			& Alta\\}

%%%%%%%%%%%%%%%%%%%%%%%%%%%%%%%%%%%%%%%%%%%%%%%%%%%%%%%%%%%%%%%%%%%%%%%%%%%%%%%%%%%%%%%%%%%%%%%%%%%%%%%%%%%%%%%%%%%%%%%%%%%%%

\tablaSinColores{CU-22 Visualizar información}
{L{3.5cm} L{10cm}}
{2}
{Tabla CU-22}
{\textbf{CU-22} & \textbf{Visualizar información} \\}
{\textbf{Versión} 				& 1.0\\ 
	\textbf{Autor} 				& Juan Pedro Pascual Vitores\\
	\textbf{Requisitos asociados} 	& RF-4.2\\
	\textbf{Descripción} 			& 
	Permite al usuario visualizar la información de un edificio.\\
	\textbf{Precondiciones} 		& 
	\begin{itemize}
		\item Se encuentran disponibles las bases de datos.
		\item Se encuentran disponibles las REST API.
		\item Se accede como usuario Viewer (no autenticado).
		\item Hay edificios creados.
	\end{itemize}
	\\
	\textbf{Acciones} 				&  
	\begin{enumerate}
		\item El usuario accede a la herramienta Viewer.
		\item El usuario pincha sobre clusters que aparecen en el plano de localización hasta ver un icono ``G'', el cual significa que, ahí se encuentra un edificio.
		\item El usuario pincha sobre el icono ``G'' y se debe visualizar toda la información del edificio, número de planta, el plano del edificio, los pois, sus rutas predefinidas etc\ldots. Es decir debe ver toda la información que el usuario administrador (Architect) desee mostrar.
	\end{enumerate}
	\\
	
	\textbf{Postcondiciones} 		& 
	\begin{itemize}
		\item La información del edificio se ve tal como el usuario administrador (Architect), la ha introducido.
	\end{itemize}
	\\
	\textbf{Excepciones} 			& 
	\begin{itemize}
		\item Las bases de datos no están disponibles.
		\item Las REST API no está 
		\item No existen edificios.
	\end{itemize}
	
	\\
	\textbf{Importancia} 			& Alta\\}

%%%%%%%%%%%%%%%%%%%%%%%%%%%%%%%%%%%%%%%%%%%%%%%%%%%%%%%%%%%%%%%%%%%%%%%%%%%%%%%%%%%%%%%%%%%%%%%%%%%%%%%%%%%%%%%%%%%%%%%%%%%%%

\tablaSinColores{CU-23 Visualización de pois}
{L{3.5cm} L{10cm}}
{2}
{Tabla CU-23}
{\textbf{CU-23} & \textbf{Visualización de pois} \\}
{\textbf{Versión} 				& 1.0\\ 
	\textbf{Autor} 				& Juan Pedro Pascual Vitores\\
	\textbf{Requisitos asociados} 	& RF-5, RF-5.1, RF-5.2, RF-5.3\\
	\textbf{Descripción} 			& 
	Permite al usuario visualizar los pois de un edificio.\\
	\textbf{Precondiciones} 		& 
	\begin{itemize}
		\item Se encuentran disponibles las bases de datos.
		\item Se encuentran disponibles las REST API.
		\item Se accede como usuario Viewer (no autenticado).
		\item Hay edificios creados y pois.
	\end{itemize}
	\\
	\textbf{Acciones} 				&  
	\begin{enumerate}
		\item El usuario accede a la herramienta Viewer.
		\item El usuario pincha sobre clusters que aparecen en el plano de localización hasta ver un icono ``G'', el cual significa que, ahí se encuentra un edificio.
		\item El usuario pincha sobre el icono ``G'' y se visualiza la información del edificio. Además visualizan los pois correspondientes al edificio y la planta.
	\end{enumerate}
	\\
	
	\textbf{Postcondiciones} 		& 
	\begin{itemize}
		\item La información de los pois del edificio se ve tal como el usuario administrador (Architect), la ha introducido.
	\end{itemize}
	\\
	\textbf{Excepciones} 			& 
	\begin{itemize}
		\item Las bases de datos no están disponibles.
		\item Las REST API no está 
		\item No existen edificios o pois.
	\end{itemize}
	
	\\
	\textbf{Importancia} 			& Alta\\}

%%%%%%%%%%%%%%%%%%%%%%%%%%%%%%%%%%%%%%%%%%%%%%%%%%%%%%%%%%%%%%%%%%%%%%%%%%%%%%%%%%%%%%%%%%%%%%%%%%%%%%%%%%%%%%%%%%%%%%%%%%%%%

\tablaSinColores{CU-24 Visualizar información}
{L{3.5cm} L{10cm}}
{2}
{Tabla CU-24}
{\textbf{CU-24} & \textbf{Visualizar información} \\}
{\textbf{Versión} 				& 1.0\\ 
	\textbf{Autor} 				& Juan Pedro Pascual Vitores\\
	\textbf{Requisitos asociados} 	& RF-5.1\\
	\textbf{Descripción} 			& 
	Permite al usuario visualizar la información de los pois un edificio.\\
	\textbf{Precondiciones} 		& 
	\begin{itemize}
		\item Se encuentran disponibles las bases de datos.
		\item Se encuentran disponibles las REST API.
		\item Se accede como usuario Viewer (no autenticado).
		\item Hay edificios creados y pois.
	\end{itemize}
	\\
	\textbf{Acciones} 				&  
	\begin{enumerate}
		\item El usuario accede a la herramienta Viewer.
		\item El usuario pincha sobre clusters que aparecen en el plano de localización hasta ver un icono ``G'', el cual significa que, ahí se encuentra un edificio.
		\item El usuario pincha sobre el icono ``G'' y se visualiza la información del edificio.
		\item El usuario pincha sobre un poi (los poi tienen iconos representativos, para identificarlos). Al pinchar en el poi aparece un panel desplegable en el que se puede visualizar su información, además pulsando en el icono ``i'' se visualizará la descripción del poi. 
	\end{enumerate}
	\\
	
	\textbf{Postcondiciones} 		& 
	\begin{itemize}
		\item La información de los pois del edificio se ve tal como el usuario administrador (Architect), la ha introducido.
	\end{itemize}
	\\
	\textbf{Excepciones} 			& 
	\begin{itemize}
		\item Las bases de datos no están disponibles.
		\item Las REST API no está 
		\item No existen edificios o pois.
	\end{itemize}
	
	\\
	\textbf{Importancia} 			& Alta\\}

%%%%%%%%%%%%%%%%%%%%%%%%%%%%%%%%%%%%%%%%%%%%%%%%%%%%%%%%%%%%%%%%%%%%%%%%%%%%%%%%%%%%%%%%%%%%%%%%%%%%%%%%%%%%%%%%%%%%%%%%%%%%%

\tablaSinColores{CU-25 Marcar un poi}
{L{3.5cm} L{10cm}}
{2}
{Tabla CU-25}
{\textbf{CU-25} & \textbf{Marcar un poi} \\}
{\textbf{Versión} 				& 1.0\\ 
	\textbf{Autor} 				& Juan Pedro Pascual Vitores\\
	\textbf{Requisitos asociados} 	& RF-5.2\\
	\textbf{Descripción} 			& 
	Permite al usuario marcar o seleccionar un poi.\\
	\textbf{Precondiciones} 		& 
	\begin{itemize}
		\item Se encuentran disponibles las bases de datos.
		\item Se encuentran disponibles las REST API.
		\item Se accede como usuario Viewer (no autenticado).
		\item Hay edificios creados y pois.
	\end{itemize}
	\\
	\textbf{Acciones} 				&  
	\begin{enumerate}
		\item El usuario accede a la herramienta Viewer.
		\item El usuario pincha sobre clusters que aparecen en el plano de localización hasta ver un icono ``G'', el cual significa que, ahí se encuentra un edificio.
		\item El usuario pincha sobre el icono ``G'' y se visualiza la información del edificio.
		\item El usuario pincha sobre un poi para marcarlo, los pois tienen iconos representativos, para identificarlos.
	\end{enumerate}
	\\
	
	\textbf{Postcondiciones} 		& 
	\begin{itemize}
		\item La información de los pois del edificio se ve tal como el usuario administrador (Architect), la ha introducido.
	\end{itemize}
	\\
	\textbf{Excepciones} 			& 
	\begin{itemize}
		\item Las bases de datos no están disponibles.
		\item Las REST API no está 
		\item No existen edificios o pois.
	\end{itemize}
	
	\\
	\textbf{Importancia} 			& Alta\\}

%%%%%%%%%%%%%%%%%%%%%%%%%%%%%%%%%%%%%%%%%%%%%%%%%%%%%%%%%%%%%%%%%%%%%%%%%%%%%%%%%%%%%%%%%%%%%%%%%%%%%%%%%%%%%%%%%%%%%%%%%%%%%

\tablaSinColores{CU-26 Proyectar camino de forma automática }
{L{3.5cm} L{10cm}}
{2}
{Tabla CU-26}
{\textbf{CU-26} & \textbf{Proyectar camino de forma automática} \\}
{\textbf{Versión} 				& 1.0\\ 
	\textbf{Autor} 				& Juan Pedro Pascual Vitores\\
	\textbf{Requisitos asociados} 	& RF-5.3\\
	\textbf{Descripción} 			& 
	Permite al usuario al marcar un poi, que se dibuje un camino automático desde la ubicación del usuario hasta el poi indicado.\\
	\textbf{Precondiciones} 		& 
	\begin{itemize}
		\item Se encuentran disponibles las bases de datos.
		\item Se encuentran disponibles las REST API.
		\item Se accede como usuario Viewer (no autenticado).
		\item Hay edificios creados y pois.
	\end{itemize}
	\\
	\textbf{Acciones} 				&  
	\begin{enumerate}
		\item El usuario accede a la herramienta Viewer.
		\item El usuario pincha sobre clusters que aparecen en el plano de localización hasta ver un icono ``G'', el cual significa que, ahí se encuentra un edificio.
		\item El usuario pincha sobre el icono ``G'' y se visualiza la información del edificio.
		\item El usuario pincha sobre un poi para marcarlo.
		\item El usuario pincha sobre el icono de navegación que aparece en el panel emergente, para proyectar el camino.
	\end{enumerate}
	\\
	
	\textbf{Postcondiciones} 		& 
	\begin{itemize}
		\item La información de los pois del edificio se ve tal como el usuario administrador (Architect), la ha introducido.
	\end{itemize}
	\\
	\textbf{Excepciones} 			& 
	\begin{itemize}
		\item Las bases de datos no están disponibles.
		\item Las REST API no está 
		\item No existen edificios o pois.
		\item No se puede localizar al usuario.
	\end{itemize}
	
	\\
	\textbf{Importancia} 			& Alta\\}

%%%%%%%%%%%%%%%%%%%%%%%%%%%%%%%%%%%%%%%%%%%%%%%%%%%%%%%%%%%%%%%%%%%%%%%%%%%%%%%%%%%%%%%%%%%%%%%%%%%%%%%%%%%%%%%%%%%%%%%%%%%%%

\tablaSinColores{CU-27 Visualización de rutas }
{L{3.5cm} L{10cm}}
{2}
{Tabla CU-27}
{\textbf{CU-27} & \textbf{Visualización de rutas} \\}
{\textbf{Versión} 				& 1.0\\ 
	\textbf{Autor} 				& Juan Pedro Pascual Vitores\\
	\textbf{Requisitos asociados} 	& RF-6, RF-6.1, RF-6.2, RF-6.3 \\
	\textbf{Descripción} 			& 
	Permite al usuario visualizar las rutas predefinidas del edificio seleccionado.\\
	\textbf{Precondiciones} 		& 
	\begin{itemize}
		\item Se encuentran disponibles las bases de datos.
		\item Se encuentran disponibles las REST API.
		\item Se accede como usuario Viewer (no autenticado).
		\item Hay edificios creados y rutas predefinidas.
	\end{itemize}
	\\
	\textbf{Acciones} 				&  
	\begin{enumerate}
		\item El usuario accede a la herramienta Viewer.
		\item El usuario pincha sobre clusters que aparecen en el plano de localización hasta ver un icono ``G'', el cual significa que, ahí se encuentra un edificio.
		\item El usuario pincha sobre el icono ``G'' y se visualiza la información del edificio.
		\item El usuario pincha sobre el desplegable que aparece arriba a la derecha para seleccionar las rutas predefinidas del edificio.
	\end{enumerate}
	\\
	
	\textbf{Postcondiciones} 		& 
	\begin{itemize}
		\item La información de las rutas predefinidas del edificio se ve tal como el usuario administrador (Architect), la ha introducido.
	\end{itemize}
	\\
	\textbf{Excepciones} 			& 
	\begin{itemize}
		\item Las bases de datos no están disponibles.
		\item Las REST API no está 
		\item No existen edificios o rutas predefinidas.
	\end{itemize}
	
	\\
	\textbf{Importancia} 			& Alta\\}

%%%%%%%%%%%%%%%%%%%%%%%%%%%%%%%%%%%%%%%%%%%%%%%%%%%%%%%%%%%%%%%%%%%%%%%%%%%%%%%%%%%%%%%%%%%%%%%%%%%%%%%%%%%%%%%%%%%%%%%%%%%%%

\tablaSinColores{CU-28 Seleccionar ruta }
{L{3.5cm} L{10cm}}
{2}
{Tabla CU-28}
{\textbf{CU-28} & \textbf{Seleccionar ruta} \\}
{\textbf{Versión} 				& 1.0\\ 
	\textbf{Autor} 				& Juan Pedro Pascual Vitores\\
	\textbf{Requisitos asociados} 	& RF-6.1 \\
	\textbf{Descripción} 			& 
	Permite al usuario elegir entre las diferentes rutas predefinidas existentes en un edificio.\\
	\textbf{Precondiciones} 		& 
	\begin{itemize}
		\item Se encuentran disponibles las bases de datos.
		\item Se encuentran disponibles las REST API.
		\item Se accede como usuario Viewer (no autenticado).
		\item Hay edificios creados y rutas predefinidas.
	\end{itemize}
	\\
	\textbf{Acciones} 				&  
	\begin{enumerate}
		\item El usuario accede a la herramienta Viewer.
		\item El usuario pincha sobre clusters que aparecen en el plano de localización hasta ver un icono ``G'', el cual significa que, ahí se encuentra un edificio.
		\item El usuario pincha sobre el icono ``G'' y se visualiza la información del edificio.
		\item El usuario pincha sobre el desplegable que aparece arriba a la derecha para seleccionar entre las rutas predefinidas del edificio.
	\end{enumerate}
	\\
	
	\textbf{Postcondiciones} 		& 
	\begin{itemize}
		\item La información de las rutas predefinidas del edificio se ve tal como el usuario administrador (Architect), la ha introducido.
	\end{itemize}
	\\
	\textbf{Excepciones} 			& 
	\begin{itemize}
		\item Las bases de datos no están disponibles.
		\item Las REST API no está 
		\item No existen edificios o rutas predefinidas.
	\end{itemize}
	
	\\
	\textbf{Importancia} 			& Alta\\}

%%%%%%%%%%%%%%%%%%%%%%%%%%%%%%%%%%%%%%%%%%%%%%%%%%%%%%%%%%%%%%%%%%%%%%%%%%%%%%%%%%%%%%%%%%%%%%%%%%%%%%%%%%%%%%%%%%%%%%%%%%%%%

\tablaSinColores{CU-29 Mostrar ruta }
{L{3.5cm} L{10cm}}
{2}
{Tabla CU-29}
{\textbf{CU-29} & \textbf{Mostrar ruta} \\}
{\textbf{Versión} 				& 1.0\\ 
	\textbf{Autor} 				& Juan Pedro Pascual Vitores\\
	\textbf{Requisitos asociados} 	& RF-6.1 \\
	\textbf{Descripción} 			& 
	Permite al usuario ver el trazado de la ruta.\\
	\textbf{Precondiciones} 		& 
	\begin{itemize}
		\item Se encuentran disponibles las bases de datos.
		\item Se encuentran disponibles las REST API.
		\item Se accede como usuario Viewer (no autenticado).
		\item Hay edificios creados y rutas predefinidas.
	\end{itemize}
	\\
	\textbf{Acciones} 				&  
	\begin{enumerate}
		\item El usuario accede a la herramienta Viewer.
		\item El usuario pincha sobre clusters que aparecen en el plano de localización hasta ver un icono ``G'', el cual significa que, ahí se encuentra un edificio.
		\item El usuario pincha sobre el icono ``G'' y se visualiza la información del edificio.
		\item El usuario pincha sobre el desplegable que aparece arriba a la derecha. 
		\item El usuario selecciona una ruta predefinida, lo que hará que se muestre su trazado.
	\end{enumerate}
	\\
	
	\textbf{Postcondiciones} 		& 
	\begin{itemize}
		\item La información de las rutas predefinidas del edificio se ve tal como el usuario administrador (Architect), la ha introducido.
	\end{itemize}
	\\
	\textbf{Excepciones} 			& 
	\begin{itemize}
		\item Las bases de datos no están disponibles.
		\item Las REST API no está 
		\item No existen edificios o rutas predefinidas.
	\end{itemize}
	
	\\
	\textbf{Importancia} 			& Alta\\}

%%%%%%%%%%%%%%%%%%%%%%%%%%%%%%%%%%%%%%%%%%%%%%%%%%%%%%%%%%%%%%%%%%%%%%%%%%%%%%%%%%%%%%%%%%%%%%%%%%%%%%%%%%%%%%%%%%%%%%%%%%%%%

\tablaSinColores{CU-30 Limpiar ruta }
{L{3.5cm} L{10cm}}
{2}
{Tabla CU-30}
{\textbf{CU-30} & \textbf{Limpiar ruta} \\}
{\textbf{Versión} 				& 1.0\\ 
	\textbf{Autor} 				& Juan Pedro Pascual Vitores\\
	\textbf{Requisitos asociados} 	& RF-6.1 \\
	\textbf{Descripción} 			& 
	Permite al usuario dejar de ver el trazado de la ruta.\\
	\textbf{Precondiciones} 		& 
	\begin{itemize}
		\item Se encuentran disponibles las bases de datos.
		\item Se encuentran disponibles las REST API.
		\item Se accede como usuario Viewer (no autenticado).
		\item Hay edificios creados y rutas predefinidas.
	\end{itemize}
	\\
	\textbf{Acciones} 				&  
	\begin{enumerate}
		\item El usuario accede a la herramienta Viewer.
		\item El usuario pincha sobre clusters que aparecen en el plano de localización hasta ver un icono ``G'', el cual significa que, ahí se encuentra un edificio.
		\item El usuario pincha sobre el icono ``G'' y se visualiza la información del edificio.
		\item El usuario pincha sobre el desplegable que aparece arriba a la derecha. 
		\item El usuario selecciona una ruta predefinida, lo que hará que se muestre su trazado.
		\item El usuario pincha sobre el botón ``x'' para dejar de ver el trazado de la ruta predefinida.
	\end{enumerate}
	\\
	
	\textbf{Postcondiciones} 		& 
	\begin{itemize}
		\item La información de las rutas predefinidas del edificio se ve tal como el usuario administrador (Architect), la ha introducido.
	\end{itemize}
	\\
	\textbf{Excepciones} 			& 
	\begin{itemize}
		\item Las bases de datos no están disponibles.
		\item Las REST API no está 
		\item No existen edificios o rutas predefinidas.
	\end{itemize}
	
	\\
	\textbf{Importancia} 			& Media\\}

%%%%%%%%%%%%%%%%%%%%%%%%%%%%%%%%%%%%%%%%%%%%%%%%%%%%%%%%%%%%%%%%%%%%%%%%%%%%%%%%%%%%%%%%%%%%%%%%%%%%%%%%%%%%%%%%%%%%%%%%%%%%%