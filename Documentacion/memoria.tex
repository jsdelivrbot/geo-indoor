\documentclass[a4paper,11pt,oneside]{memoir}

% Castellano
\usepackage[spanish,es-tabla]{babel}
\selectlanguage{spanish}
\usepackage[utf8]{inputenc}
\usepackage{placeins}

\RequirePackage{booktabs}
\RequirePackage[table]{xcolor}
\RequirePackage{xtab}
\RequirePackage{multirow}

% Verde oscuro
\definecolor{darkgreen}{rgb}{0.0, 0.35, 0.0}

% Links
\usepackage[colorlinks]{hyperref}
\hypersetup{
	allcolors = {darkgreen}
}

%Citas de bibliografía
\usepackage{cite}
\usepackage[numbers,sort]{natbib}

% Ecuaciones
\usepackage{amsmath}

% Rutas de fichero / paquete
\newcommand{\ruta}[1]{{\sffamily #1}}

% Párrafos
\nonzeroparskip


% Imagenes
\usepackage{graphicx}
\newcommand{\imagen}[2]{
	\begin{figure}[!h]
		\centering
		\includegraphics[width=0.9\textwidth]{#1}
		\caption{#2}\label{fig:#1}
	\end{figure}
	\FloatBarrier
}

\newcommand{\imagenflotante}[2]{
	\begin{figure}%[!h]
		\centering
		\includegraphics[width=0.9\textwidth]{#1}
		\caption{#2}\label{fig:#1}
	\end{figure}
}

%COMANDOS GEOINDOOR -- START --

% El comando \imagen_resize nos permite insertar figuras ajustando el tamaño manualmente
\newcommand{\imagenResize}[4]{
	\begin{figure}[h!]
		\centering
		\includegraphics[scale= #1  ]{#2}\label{fig:#4}
		\caption{#3}
	\end{figure}
}

%Comando para añadir espacios de 1cm, como tabulador
\newcommand\tab[1][1cm]{
	\hspace*{#1}
}

%COMANDOS GEOINDOOR -- END --

% El comando \figura nos permite insertar figuras comodamente, y utilizando
% siempre el mismo formato. Los parametros son:
% 1 -> Porcentaje del ancho de página que ocupará la figura (de 0 a 1)
% 2 --> Fichero de la imagen
% 3 --> Texto a pie de imagen
% 4 --> Etiqueta (label) para referencias
% 5 --> Opciones que queramos pasarle al \includegraphics
% 6 --> Opciones de posicionamiento a pasarle a \begin{figure}
\newcommand{\figuraConPosicion}[6]{%
  \setlength{\anchoFloat}{#1\textwidth}%
  \addtolength{\anchoFloat}{-4\fboxsep}%
  \setlength{\anchoFigura}{\anchoFloat}%
  \begin{figure}[#6]
    \begin{center}%
      \Ovalbox{%
        \begin{minipage}{\anchoFloat}%
          \begin{center}%
            \includegraphics[width=\anchoFigura,#5]{#2}%
            \caption{#3}%
            \label{#4}%
          \end{center}%
        \end{minipage}
      }%
    \end{center}%
  \end{figure}%
}

%
% Comando para incluir imágenes en formato apaisado (sin marco).
\newcommand{\figuraApaisadaSinMarco}[5]{%
  \begin{figure}%
    \begin{center}%
    \includegraphics[angle=90,height=#1\textheight,#5]{#2}%
    \caption{#3}%
    \label{#4}%
    \end{center}%
  \end{figure}%
}
% Para las tablas
\newcommand{\otoprule}{\midrule [\heavyrulewidth]}
%
% Nuevo comando para tablas pequeñas (menos de una página).
\newcommand{\tablaSmall}[5]{%
 \begin{table}[h!] %MODIFICACIÓN AÑADIDA -> [h!]
  \begin{center}
   \rowcolors {2}{gray!35}{}
   \begin{tabular}{#2}
    \toprule
    #4
    \otoprule
    #5
    \bottomrule
   \end{tabular}
   \caption{#1}
   \label{tabla:#3}
  \end{center}
 \end{table}
}

%
% Nuevo comando para tablas pequeñas (menos de una página).
\newcommand{\tablaSmallSinColores}[5]{%
 \begin{table}[H]
  \begin{center}
   \begin{tabular}{#2}
    \toprule
    #4
    \otoprule
    #5
    \bottomrule
   \end{tabular}
   \caption{#1}
   \label{tabla:#3}
  \end{center}
 \end{table}
}

\newcommand{\tablaApaisadaSmall}[5]{%
\begin{landscape}
  \begin{table}
   \begin{center}
    \rowcolors {2}{gray!35}{}
    \begin{tabular}{#2}
     \toprule
     #4
     \otoprule
     #5
     \bottomrule
    \end{tabular}
    \caption{#1}
    \label{tabla:#3}
   \end{center}
  \end{table}
\end{landscape}
}

%
% Nuevo comando para tablas grandes con cabecera y filas alternas coloreadas en gris.
\newcommand{\tabla}[6]{%
  \begin{center}
    \tablefirsthead{
      \toprule
      #5
      \otoprule
    }
    \tablehead{
      \multicolumn{#3}{l}{\small\sl continúa desde la página anterior}\\
      \toprule
      #5
      \otoprule
    }
    \tabletail{
      \hline
      \multicolumn{#3}{r}{\small\sl continúa en la página siguiente}\\
    }
    \tablelasttail{
      \hline
    }
    \bottomcaption{#1}
    \rowcolors {2}{gray!35}{}
    \begin{xtabular}{#2}
      #6
      \bottomrule
    \end{xtabular}
    \label{tabla:#4}
  \end{center}
}

%
% Nuevo comando para tablas grandes con cabecera.
\newcommand{\tablaSinColores}[6]{%
  \begin{center}
    \tablefirsthead{
      \toprule
      #5
      \otoprule
    }
    \tablehead{
      \multicolumn{#3}{l}{\small\sl continúa desde la página anterior}\\
      \toprule
      #5
      \otoprule
    }
    \tabletail{
      \hline
      \multicolumn{#3}{r}{\small\sl continúa en la página siguiente}\\
    }
    \tablelasttail{
      \hline
    }
    \bottomcaption{#1}
    \begin{xtabular}{#2}
      #6
      \bottomrule
    \end{xtabular}
    \label{tabla:#4}
  \end{center}
}

%
% Nuevo comando para tablas grandes sin cabecera.
\newcommand{\tablaSinCabecera}[5]{%
  \begin{center}
    \tablefirsthead{
      \toprule
    }
    \tablehead{
      \multicolumn{#3}{l}{\small\sl continúa desde la página anterior}\\
      \hline
    }
    \tabletail{
      \hline
      \multicolumn{#3}{r}{\small\sl continúa en la página siguiente}\\
    }
    \tablelasttail{
      \hline
    }
    \bottomcaption{#1}
  \begin{xtabular}{#2}
    #5
   \bottomrule
  \end{xtabular}
  \label{tabla:#4}
  \end{center}
}



\definecolor{cgoLight}{HTML}{EEEEEE}
\definecolor{cgoExtralight}{HTML}{FFFFFF}

%
% Nuevo comando para tablas grandes sin cabecera.
\newcommand{\tablaSinCabeceraConBandas}[5]{%
  \begin{center}
    \tablefirsthead{
      \toprule
    }
    \tablehead{
      \multicolumn{#3}{l}{\small\sl continúa desde la página anterior}\\
      \hline
    }
    \tabletail{
      \hline
      \multicolumn{#3}{r}{\small\sl continúa en la página siguiente}\\
    }
    \tablelasttail{
      \hline
    }
    \bottomcaption{#1}
    \rowcolors[]{1}{cgoExtralight}{cgoLight}

  \begin{xtabular}{#2}
    #5
   \bottomrule
  \end{xtabular}
  \label{tabla:#4}
  \end{center}
}


















\graphicspath{ {./img/} }

% Capítulos
\chapterstyle{bianchi}
\newcommand{\capitulo}[2]{
	\setcounter{chapter}{#1}
	\setcounter{section}{0}
	\chapter*{#2}
	\addcontentsline{toc}{chapter}{#2}
	\markboth{#2}{#2}
}

% Apéndices
\renewcommand{\appendixname}{Apéndice}
\renewcommand*\cftappendixname{\appendixname}

\newcommand{\apendice}[1]{
	%\renewcommand{\thechapter}{A}
	\chapter{#1}
}

\renewcommand*\cftappendixname{\appendixname\ }

% Formato de portada
\makeatletter
\usepackage{xcolor}
\newcommand{\tutor}[1]{\def\@tutor{#1}}
\newcommand{\course}[1]{\def\@course{#1}}
\definecolor{cpardoBox}{HTML}{E6E6FF}
\def\maketitle{
  \null
  \thispagestyle{empty}
  % Cabecera ----------------
\noindent\includegraphics[width=\textwidth]{cabecera}\vspace{1cm}%
  \vfill
  % Título proyecto y escudo informática ----------------
  \colorbox{cpardoBox}{%
    \begin{minipage}{.8\textwidth}
      \vspace{.5cm}\Large
      \begin{center}
      \textbf{TFG del Grado en Ingeniería Informática}\vspace{.6cm}\\
      \textbf{\LARGE\@title{}}
      \end{center}
      \vspace{.2cm}
    \end{minipage}

  }%
  \hfill\begin{minipage}{.20\textwidth}
    \includegraphics[width=\textwidth]{escudoInfor}
  \end{minipage}
  \vfill
  % Datos de alumno, curso y tutores ------------------
  % LOGO GEOINDOOR START
	\begin{figure}[h!]
		\centering
		\includegraphics[scale=1.5]{img/logo}
	\end{figure}
  % LOGO GEOINDOOR END	
  \begin{center}%
  {%
    \noindent\LARGE
    Presentado por  \@author{}\\ 
    en Universidad de Burgos ---  \@date{}\\
    Tutor:   \@tutor{Carlos López Nozal}\\
  }%
  \end{center}%
  \null
  \cleardoublepage
  }
\makeatother

\newcommand{\nombre}{Juan Pedro Pascual Vitores} %%% cambio de comando

% Datos de portada
\title{Geoindoor}
\author{\nombre}
\tutor{}
\date{\today}

\begin{document}

\maketitle


\newpage\null\thispagestyle{empty}\newpage


%%%%%%%%%%%%%%%%%%%%%%%%%%%%%%%%%%%%%%%%%%%%%%%%%%%%%%%%%%%%%%%%%%%%%%%%%%%%%%%%%%%%%%%%
\thispagestyle{empty}


\noindent\includegraphics[width=\textwidth]{cabecera}\vspace{1cm}

\noindent D. Carlos López Nozal, profesor del departamento de ingeniería civil , área de lenguaje de sistemas.

\noindent Expone:

\noindent Que el alumno D. \nombre, con DNI 13172380G, ha realizado el Trabajo final de Grado en Ingeniería Informática titulado Geoindoor de TFG. 

\noindent Y que dicho trabajo ha sido realizado por el alumno bajo la dirección del que suscribe, en virtud de lo cual se autoriza su presentación y defensa.

\begin{center} %\large
En Burgos, {\large \today}
\end{center}

\vfill\vfill\vfill

% Author and supervisor
\begin{minipage}{0.45\textwidth}
\begin{flushleft} %\large
Vº. Bº. del Tutor:\\[2cm]
D. Carlos López Nozal
\end{flushleft}
\end{minipage}


\vfill

% para casos con solo un tutor comentar lo anterior
% y descomentar lo siguiente
%Vº. Bº. del Tutor:\\[2cm]
%D. nombre tutor


\newpage\null\thispagestyle{empty}\newpage




\frontmatter

% Abstract en castellano
\renewcommand*\abstractname{Resumen}
\begin{abstract}
La geolocalización es un concepto que se lleva estudiando desde hace años, pero con el avance de la tecnología se ha conseguido acercar la geolocalización al ámbito cotidiano, consiguiendo que nos resulte más fácil llegar a un determinado lugar.

Habitualmente se ha utilizado la geolocalización para orientarnos en extensiones relativamente amplias. Sin embargo el avance del hardware permite que la geolocalización sea más precisa, con lo que están apareciendo herramientas para que la geolocalización sea aplicada a extensiones de terreno más pequeñas, abriendo un mundo de posibilidades, y precisamente sobre esto trata el proyecto.

Geoindoor es una herramienta con múltiples aplicaciones, que añade servicios de geolocalización dentro edificios. Una aplicación Web permite etiquetar ubicaciones dentro del plano del edificio. Otra aplicación permite consultar las ubicaciones de un edificio , las posiciones en plano del edificio y navegar hasta ellas desde la ubicación actual.
\end{abstract}

\renewcommand*\abstractname{Descriptores}
\begin{abstract}
Geolocalización indoor, sistema de localización, búsqueda de lugares, localización en interiores, rutas predeterminadas. \ldots
\end{abstract}

\clearpage

% Abstract en inglés
\renewcommand*\abstractname{Abstract}
\begin{abstract}
Geolocation is a concept that has been studied since years, but the advancement of technology has been able to bring geolocation closer to us, because of this we arrive to the places faster.

Usually geolocation has been used to orient in large extensions. However, the hardware advancement allows the geolocation to be more accurate, because of this, there are more tools for the geolocation dedicated to smaller terrain extensions, opening a world of possibilities and the project is about it.

Geoindoor is a tool with multiple applications that add geolocation services within buildings. A web application allows you to tag locations within the building plan. Another application that allows you to consult the locations of a building, positions in the building plan and navigate to them from the actual location.
\end{abstract}

\renewcommand*\abstractname{Keywords}
\begin{abstract}
Indoor geolocation, location system, location search, indoor location, predetermined routes. \ldots
\end{abstract}

\clearpage

% Indices
\tableofcontents

\clearpage

\listoffigures

\clearpage

\listoftables
\clearpage

\mainmatter
\capitulo{1}{Introducción}

\texttt{"No se dónde está"}\\
\\
Es habitual escuchar esta frase, que multitud de veces irrumpe en nuestras vidas dificultando nuestras tareas, tareas
tan simples como encontrar un baño, una oficina, o la cafetería. Estos son los típicos problemas que se encuentra uno 
cuando acude por primera vez a un lugar, pero es más grave para aquel que ofrece un servicio o unos productos en un 
determinado lugar y el usuario no lo localiza. Esto puede hacer que el flujo de personas que acudan al establecimiento 
sea mucho menor, lo que provocaría que el servicio estuviera poco solicitado o que los productos ofrecidos no fuesen vistos 
por tantas personas como se desearía. Es aún más grave si hablamos de un negocio el cual necesite que los usuarios localicen
con rapidez donde se ofrecen los servicios o productos. 
 
De ahí la necesidad de la implementación del proyecto, que propone dar servicios de geolocalización indoor. Esto resolvería 
diversos problemas en establecimientos como museos, exposiciones, centros comerciales, hospitales,
o cualquier otro establecimiento de gran tamaño.

La geolocalización indoor además de servir para encontrar localizaciones determinadas, puede ser aplicada para programar 
rutas que se amolden a unos intereses, y así hacer de la experiencia de moverse por nuevos lugares algo más fructífero y gratificante.

\section{Estructura de la memoria}\label{estructura-de-la-memoria}

La memoria tiene la siguiente estructura:

\begin{itemize}
\tightlist
\item
  \textbf{Introducción:} exposición del problema a resolver y como lo resuelve el proyecto.
   Estructura de la memoria y listado de materiales
  adjuntos.
\item
  \textbf{Objetivos del proyecto:} objetivos del proyecto.
\item
  \textbf{Conceptos teóricos:} explicación de los conceptos teóricos necesarios para comprender el proyecto.
\item
  \textbf{Técnicas y herramientas:} técnicas y herramientas utilizadas para la realización de proyecto.
\item
  \textbf{Aspectos relevantes del desarrollo:} explicación de dificultades y obstáculos encontrados en la realización del proyecto.
\item
  \textbf{Trabajos relacionados:} trabajos relacionados con el proyecto.
\item
  \textbf{Conclusiones y líneas de trabajo futuras:} conclusiones
  obtenidas tras realizar el proyecto y expectativas de futuro.
\end{itemize}

Junto a la memoria se proporcionan los siguientes anexos:

\begin{itemize}
\tightlist
\item
  \textbf{Plan del proyecto software:} sección centrada planificación temporal y viabilidad del proyecto.
\item
  \textbf{Especificación de requisitos del software:} sección que se centra en la especificación de requisitos.
\item
  \textbf{Especificación de diseño:} sección que se basa en describir la fase de diseño.
\item
  \textbf{Manual del programador:} sección centrada en la explicación del código fuente (estructura, compilación,
  instalación, ejecución, pruebas,etc... ).
\item
  \textbf{Manual de usuario:} sección centrada en la explicación al usuario de como utilizar la herramienta.
\end{itemize}

\section{Materiales adjuntos}\label{materiales-adjuntos}

Los materiales adjuntados con la memoria son: 

\begin{itemize}
\tightlist
\item
	Architect, parte de la herramienta para crear rutas y edificios.
\item
	Viewer, parte de la herramienta encargada de la visualización .
\item	
	Código fuente de la Rest API encargada de interactuar con la BD.
\item	
	Pruebas de integración.
\item	
	Pruebas de estrés y rendimiento.
\end{itemize}

Recursos en internet:

\begin{itemize}
\tightlist
\item
  Architect (Se lanza en el localhost).
\item
  Viewer (Se lanza en el localhost).
\item
  Rest API https://geoindoorapi.herokuapp.com/ .
\item
  Base de datos de firebase  https://geoindoordb.firebaseio.com/ .
\item
  Repositorio del proyecto .
\end{itemize}

\capitulo{2}{Objetivos del proyecto}

A continuación, se especifican los objetivos que promueven la realización
de este proyecto.
\section{Objetivos generales}\label{objetivos-generales}
\begin{itemize}
\item
Desarrollar una herramienta que permita añadir servicios de geolocalización
en interiores.
\item
Facilitar la localización de lugares dentro de edificios.
\item
Creación de rutas predeterminadas que permitan al usuario tanto diseñar rutas para él mismo,
como para todo el público.
\section{Objetivos técnicos}\label{objetivos-tecnicos}
\item
Uso del API proporcionado por la aplicación \textit{Anyplace}
\item
Uso de Git como sistema de control de versiones junto con la plataforma GitHub. 
\item
Uso de la metodología Scrum.
\item 
Uso de ZenHub como herramienta de gestión de proyectos.
\item
Uso de JS en la parte del servidor (Node.js).
\item
Uso de bases de datos no sql.

\section{Objetivos personales}\label{objetivos-personales}
\item
Entender cómo se realiza un proyecto desde su inicio hasta su final.
\item
Mejorar mis conocimientos y conceptos sobre el desarrollo.
\end{itemize}

\capitulo{3}{Conceptos teóricos}

A continuación se explicaran unos conceptos indispensables para el entendimiento del proyecto.
\section{Geolocalización}\label{geolocalizacion}
La Geolocalización es le concepto mas importante en este proyecto, y primero hablaremos un poco del API utilizada y después profundizaremos más en el concepto.

La API utilizada para este proyecto se basa en la API Google Maps, que está escrita en JavaScript \cite{noauthor_google_nodate:a}.

La clase principal es google.maps.Map, cuyo constructor nos permite crear un mapa dentro de un contenedor HTML. Después encontramos google.maps.LatLng, donde LatLng es un punto en coordenadas geográficas: latitud y longitud.

Para más información se puede visitar la documentación de Google Maps donde aparece información más detallada, ya que no profundizaremos en cuales, y como son las clases y funciones de esta API, si no como funciona Google Maps en si.

Google Maps funciona a través de conexión a internet y la función de GPS del dispositivo, las coordenadas de Google Maps están en el sistema WGS842 \footnote{WGS84 \textit{es un sistema de coordenadas geográficas mundial que permite localizar cualquier punto de la Tierra (sin necesitar otro de referencia) por medio de tres unidades dadas.}} y se mostrará la latitud y la longitud, positiva para Norte y Este, negativa para Sur y Oeste.

Cuando nos referimos a la función GPS, estamos hablando de la función de geolocalización del dispositivo, en la que profundizaremos con más detalle.
\\

\textbf{Definición:} Geolocalización (Geo-localización) está formada por la palabra geo, que tiene origen griego en geos que significa tierra. Y la palabra localización, de las raíces latinas localis  (relativo a lugar), que proviene de locus (lugar).

Por lo tanto, la geolocalización consiste en la ubicación de un ``lugar`` en la ``tierra`` \cite{noauthor_diccionario_nodate:a}. 

Haciendo una definición más exacta, la geolocalización consiste en ubicar o determinar donde se encuentra un objeto o lugar en un determinado espacio.

Cuando nos referimos a objeto no tiene porqué ser inanimado, y el determinado espacio al que nos referimos es la tierra como tal, pero no siempre se la pone como plano de referencia. 

Cuando hablamos sobre la geolocalización también debemos hablar de los sistemas de posicionamiento que son los encargados de analizar, manejar y procesar la información geográfica, que nos permite obtener la localización de un objeto u lugar.

El sistema de posicionamiento más conocido en el mundo es el GPS (Global Positioning System ), que es un sistema que permite localizar objetos móviles, gracias a la recepción de señales emitidas por un conjunto de satélites. Para determinar las localizaciones este sistema se sirve de 24 satélites y trilateración. Y es de origen estadounidense, otros sistemas de posicionamiento conocidos son el GLONASS, de origen ruso y Galileo europeo \cite{noauthor_sistema_2017:a}.

La trilateración es un método matemático para determinar las posiciones relativas de los objetos usando la geometría de triángulos y la triangulación. La trilateración para ubicar el objeto usa las posiciones conocidas de dos o más puntos de referencia, y la distancia entre el objeto y cada punto de referencia \cite{noauthor_trilateracion_2015:a}.


Se consigue saber la distancia entre objetos y puntos de referencia gracias a que el dispositivo emite una señal y espera una respuesta de los puntos de referencia, el diferencial de tiempo entre el envío y la recepción se utiliza para determinar la distancia.

Como se muestra en las imágenes en la trilateracion en tres dimensiones, los puntos de referencia (satélites) crean una esfera virtual o imaginaria de radio igual a la distancia entre el objeto y el satélite, donde ellos son el epicentro. Se utilizan los puntos de intersección entre esferas para calcular la posición del objeto a localizar.

Los satélites tienen que saber la distancia entre ellos por lo tanto también mandan señales, además para hacer un cálculo correcto de distancias se necesitan tener relojes coordinados.
\\
\imagen{img/GPS-Trilateracion}{Explicacion de trilateración.}



\imagen{img/trilateracion}{Explicacion de trilateración. \cite{abato_espanol:_2012:a}}


La geolocalización por wifi funciona de forma parecida a la de los sistemas de posicionamiento con satélites. Esta vez los puntos de referencia en vez de ser satélites son puntos de acceso wi-fi \cite{luz_wi-fi_2017:a}, y hay varias técnicas y tecnologías para hallar la ubicación de los dispositivos, además de la explicada anteriormente. Una de estas técnicas es la de utilizar RSSI, que indica la intensidad de la señal, a mayor intensidad más cerca está del punto de acceso Wi-Fi que proporciona la señal.  Pero normalmente la para la ubicación en interiores se utilizan ondas de radio ya que obtiene mayor precisión \cite{noauthor_mejores_nodate:a}. 

\section{Formatos gráficos escalables}\label{formatos-graficos-escalables}


Cuando hablamos de formatos gráficos escalables, nos estamos refiriendo a aquellos formatos digitales basados en objetos geométricos, cuya labor es hacer que la imagen tenga formas m\'{a}s definidas.
Los formatos más conocidos son SVG \cite{noauthorcursonodate:a}, WMF \cite{noauthor_metaarchivo_2016:a}, ODG \cite{noauthor_definicion_nodate:c} y AI \cite{noauthorextensionnodate:b}.

Debemos entender la importancia de los formatos gráficos escalables en la mejora de imágenes para hacerlas más geométricas, y este proyecto cumplen una gran labor haciendo los planos más geométricos y entendibles.


\subsection{WMF}\label{wmf}


Es un formato de gr\'{a}ficos vectoriales creado por Microsoft, naci\'{o} a principio de los 90 y ahora est\'{a} en desuso. Un archivo WMF se utiliza para regenerar una imagen a trav\'{e}s Windows GDI \footnote{Graphics Device Interface \textit{interfaz de programaci\'{o}n de aplicaciones que se encarga del control gr\'{a}fico de los dispositivos de salida}} \hypertarget{_Hlk482387012}{. WMF guarda una serie de llamadas a funciones que son enviadas a Windows GDI para regenerar la imagen.}

Enhanced Metafile (EMF) es una versi\'{o}n de 32 bits de WMF (16 bits), con comandos adicionales.



\subsection{ODG}\label{ODG}


ODG formato de imagen vectorial enlazado con la versión 2 de OpenDocument de OpenOffice, es un estándar abierto para la creación de dibujos vectoriales, utiliza XML, fue creado por Sun Microsystems y OASIS.





\subsection{AI}\label{ai}

Formato de imagen vectorial de Adobe Illustrator, está desarrollado por Adobe Systems para la representación de gráficos vectoriales en formato EPS o PDF. El formato AI está compuesto por rutas conectadas mediante puntos, en lugar de datos de imagen. 
En principio el formato AI fue un formato nativo llamado PGF hasta que se compatibilizo con PDF mediante la copia completa de datos PGF en el archivo de formato PDF.




\subsection{SVG}\label{svg}


SVG (Scalable Vector Graphics) \cite{lasso_formatos_2015:a} es un formato de imagen vectorial recomendado y desarrollado por W3C (World Wide Web Consortium), es un estándar abierto, trabaja con gráficos vectoriales bidimensionales estáticos y animados, en formato XML \cite{noauthor_scalable_2017:a}.
SVG aparece de forma nativa en multitud de navegadores como Amaya, Mozilla Firefox, Google Chrome, Safari e Internet Explorer. Para navegadores que no tienen SVG de forma nativa existen plugins que permiten mostrar imágenes en formato SVG.
SVG permite tres tipos de objetos graficos:




\begin{itemize}
\item 
	Elementos geom\'{e}tricos vectoriales



\begin{itemize}
\item 
	Rectas 



\item 
	curvas
\end{itemize}


\item 
	Mapa de bits



\item 
	Texto



\end{itemize}


Los objetos gráficos con los que trabaja SVG pueden transformarse y agruparse para después ser compuestos en otros objetos renderizados con anterioridad. 
El dibujado con SVG es dinámico e interactivo, mediante el uso de DOM para SVG (Document Object Model), que mediante ECMAScript \footnote{ECMAScript \textit{es una especificación de lenguaje de programación publicada por ECMA International. Se empezó a desarrollar en 1996 y se basó en JavaScript, fue propuesto como estándar por Netscape Communications Corporation. Actualmente es el estándar ISO 16262.}} o SMIL \footnote{SMIL (Synchronized Multimedia Integration Language)\textit{es un estándar del World Wide Web Consortium (W3C) para presentaciones multimedia, permite integrar audio, video, imágenes, texto o cualquier otro contenido multimedia.}} permite animaciones de gráficos vectoriales sencillas y eficientes
SVG tiene una gran cantidad de manejadores de eventos como ``onMouseOver" y ``onClick", que pueden ser asignados a cualquier objeto SVG. La compatibilidad de SVG con estándares web permiten aplicar características como el scripting sobre elementos SVG o XML, de forma simultánea, en la misma página web desde distintos espacios de nombre XML.
Las imágenes SVG pueden ser comprimidas con gzip (SVGZ)

\imagen{img/mapabits}{explicación diferencia mapa de bits y gráfico vectorial}

\section{Lenguajes de etiquetas}\label{lenguajes-etiuetas}
\newcommand\tab[1][1cm]{\hspace*{#1}}

Los lenguajes de etiquetas, o de marcado son utilizados para codificar un documento añadiendo etiquetas al texto, las cuales permiten otorgarle unas características o cualidades específicas. En este proyecto es mu importante ya que facilita tanto el procesado de la información, como a la hora de mostrar la información \cite{noauthor_lenguaje_2017:a}.

No se debe confundir los lenguajes de etiquetas con los de programación, ya que el lenguaje de etiquetas no tiene funciones aritméticas ni variables.

Podemos diferenciar entre 4 tipos de lenguajes de marcado \cite{noauthor_1.1_nodate:a} o de etiquetas \cite{noauthor_lenguaje_nodate:a}.
\begin{itemize}
\item
Marcado de presentación
\item
Marcado de procedimientos
\item
Marcado descriptivo
\item
Lenguajes especializados

\end{itemize}

También existen lenguajes de etiquetado que son difíciles de enmarcar en uno de estos 4 tipos, ya que comparten características de varios de ellos, como podría ser el caso de HTML, que contiene etiquetas procedimentales, como la B de bold , con otras descriptivas como es BLOCKQUOTE y atributo HREF.


(Los ejemplos representan como se suele ver el lenguaje de marcado)

\subsection{Marcado de presentación}\label{marcado-presentacion}

Indica el formato del texto, y como su propio nombre indica es utilizado para cambiar la presentación de un texto o documento. En este tipo las etiquetas suelen estar ocultas al usuario.
\\
\\
Ejemplo: Microsoft Word
\\
\\
Cuantas veces habremos oído esa frase...
\\
En muchas ocasiones nos encontramos en sitios nuevos

\subsection{Marcado de procedimientos}\label{marcado-procedimientos}

El lenguaje de marcado de procedimientos también es utilizado para la presentación del texto, pero en este caso las etiquetas sí que son visibles para los usuarios que editan el texto, y se realiza un procesamiento dependiendo de la etiqueta asociada.
\\
\\
Ejemplo: LaTeX
\\
\\
\textbackslash documentclass$[$10pt,a4paper$]$\{article\}
\\
\textbackslash usepackage$[$latin1$]$\{inputenc\}
\\
\textbackslash author\{Juan Pedro Pascual Vitores\}
\\
\textbackslash title\{Introducción\}
\\
\textbackslash begin\{document\}
\\
\textbackslash section\{Introducción\}\textbackslash label{introduccion}
\\
\textbackslash texttt\{``No se dónde está"\}
\\
Cuantas veces habremos oído esa frase...
\\
En muchas ocasiones nos encontramos en sitios nuevos
\\
\textbackslash end\{document\}
\\
\\

\subsection{Marcado descriptivo}\label{marcado-descriptivo}

Se utilizan etiquetas o marcas para describir el texto, sin especificar cómo deben ser presentados, ni en qué orden.

Este tipo de marcado tiene más usos que lo que aparenta a primera vista, pueden utilizarse para el almacenamiento de metadatos, comunicación de datos (cualquier aplicación pude escribir un documento de texto plano con datos en formato XML), migración de datos (si necesitamos migrar los datos de una base de datos a otra, si ambas trabajan en xml, el trabajo es más sencillo), almacenamiento de gráficos vectoriales (VML) \footnote{VML \textit{Vector Markup Language es un lenguaje XML destinado a la creación de los gráficos vectoriales en 2D o 3D}} \cite{noauthor_vector_2016:a}, fórmulas matemáticas con XML (MathML)\footnote{MathML \textit{Mathematical Markup Language lenguaje de marcado basado en XML \cite{noauthor_usos_2003:a}, cuyo objetivo es expresar notación matemática para que distintas máquinas puedan entenderla}}, Estructuras moleculares e información cientifica y química con XML ( CML ) \footnote{Chemical Markup Lenguaje \cite{noauthor_chemical_2016:a} \textit{es un lenguaje de marcas basado en XML cuyo objeto es expresar información molecular}} \cite{desarrolloweb.com_objetivos_nodate:b}
\\
\\
Ejemplo: XML
\\
\\
$<$?xml version=``1.0" encoding=``UTF-8" ?$>$
\\
$<$Documento$>$
\\
	 \tab $<$Texto$>$
	\\
		 \tab \tab $<$Parrafo$>$
		\\
			\tab\tab\tab Cuantas veces habremos oído esa frase...
			\\
			\tab\tab\tab En muchas ocasiones nos encontramos en sitios
			\\
		 \tab \tab $<$\textbackslash Parrafo$>$
		\\
	 \tab $<$\textbackslash Texto$>$
	\\
$<$\textbackslash Documento $>$


\subsection{Lenguajes especializados}\label{lenguajes-especializados}


Los lenguajes especializados podríamos decir que derivan de los lenguajes de marcado descriptivo, ya que se diferencian con el tipo anterior, en que se especializan en una materia como podría ser la matemática ( OpenMath o MathML ) \cite{noauthor_openmath_2017:e} \cite{noauthor_usos_2003:a} o los gráficos (SVG o VRML).
\\
\\
Ejemplo: OpenMath
\\
\\
$<$OMOBJ xmlns=``url"$>$
\\
 \tab $<$OMA cdbase=``url"$>$
 \\
    \tab\tab$<$OMS cd=``relation1" name=``eq"/$>$
    \\
   \tab\tab$<$OMV name=``x"/$>$
   \\
  \tab$<$OMA\textbackslash $>$
  \\
$<$\textbackslash OMOBJ$>$
\\
Ejemplo: SVG
\\
$<$svg width=``100" height="100"$>$
\\
  \tab$<$circle cx=``50" cy="50" r="40"\textbackslash $>$
  \\
$<$\textbackslash svg4$>$
\\





\capitulo{4}{Técnicas y herramientas}

En este apartado se hablará de las técnicas y herramientas que han sido utilizadas en el desarrollo del sistema Geoindoor. Para facilitar la comprensión de esta sección, se han organizado las herramientas según los siguientes criterios de clasificación:

\begin{itemize}
	\item \textbf{Manejadores de paquetes y Corredeores de tareas}
		\begin{itemize}
			\item Bower
			\item Grunt
		\end{itemize}
	\item \textbf{Formato de intercambio de datos}
		\begin{itemize}
			\item JSON
		\end{itemize}
	\item \textbf{Lenguajes y frameworks}
		\begin{itemize}
			\item JavaScript
			\item Python
			\item AngularJS
			\item Java (Selenium)
			\item Node.js
			\item jQuery
		\end{itemize}
	\item \textbf{Herramientas utilizadas para realizar los test}
		\begin{itemize}
			\item Selenium
			\item Webserver Stress Tool
		\end{itemize}
	\item \textbf{Software de control de versiones}
		\begin{itemize}
			\item Git
			\item ZenHub
		\end{itemize}
	\item \textbf{Documentación}
		\begin{itemize}
			\item \TeX{}
			\item \LaTeX
			\item Zotero
		\end{itemize}
	\item \textbf{Servicios web}
		\begin{itemize}
			\item Heroku
			\item Firebase
		\end{itemize}
	\item \textbf{Herramientas utilizadas para la interfaz}
		\begin{itemize}
			\item HTML5
			\item CSS3
			\item Bootstrap
		\end{itemize}
\end{itemize}

\section{Bower}\label{Bower}

Bower es un gestor de paquetes para la web, es indispensable, ya que nos ha servido para manejar los paquetes del proyecto de una forma simple y rápida. Nos ha permitido manejar las dependencias (angular, bootstrap, jquery, etc\ldots) de forma sencilla y ordenada. 

\section{Grunt}\label{Grunt}

Grunt ha sido de gran utilidad, ya que nos ha permitido compilar el proyecto de forma sencilla y rápida, ayudándonos a tener siempre centralizados los scripts y sus dependencias (imágenes, css, etc\ldots ). A su vez nos dejaba hacer cambios en los scripts en caliente y se volvía a recompilar el proyecto para poder probarlo \textit{in situ}.

Grunt es un corredor de tareas de javascript, que es utilizado para centralizar tareas y scripts, para tener que evitar procedimientos repetitivos. Si se une con Bower permite trabajar con una gran cantidad de paquetes y scripts de forma sencilla.

\section{JSON}\label{JSON}
JSON tiene un papel fundamental en el proyecto ya que al trabajar con JS se hace indispensable, además se ha utilizado en la base de datos.

JSON (JavaScript Object Notation) es un formato de intercambio de datos. Es simple, tanto a  la  hora de leerlo, como de escribirlo. Es ideal para la transmisión de objetos por la red, y es ésa la razón por la que es utilizado.

\section{JavaScript}\label{JavaScript}

JavaScript se considera el lenguaje principal del proyecto, es un lenguaje de scripting, ampliamente utilizado en todos los ámbitos de este proyecto, ya sea para cambiar el flujo de salida, recogida de datos, como para procesamiento de información y llamadas a un sevidor. Recordar que es la base de los frameworks jQuery y AngularJS y para el entorno de ejecución Node.js. Ha sido elegido por su versatilidad y capacidad de adaptarse a cualquier entorno.

\section{Python}\label{python}

Python es un lenguaje de programación interpretado que busca ser un código legible. Es un lenguaje de programación multiparadigma, programación imperativa y programación funcional.

En este proyecto es indispensable ya que es utilizado a la hora de lanzar la aplicación, permitiendo recrear un servidor de forma local.

\section{AngularJS}\label{AngularJS}

AngularJS (Google) es un framework JavaScript de desarrollo de aplicaciones web en el lado cliente, y utiliza el patrón MVC (Model-View-Controller).

AngularJS adquiere las buenas características que tiene JavaScript, pero añade que permite procesar datos de forma más sencilla y realizar un gran control sobre el flujo de salida.

Este framework se ha utilizado enormemente en este proyecto, ha sido fundamental ya que con ello hemos creado los controladores que permiten interactuar con la información y los usuarios, haciendo control sobre accesos y lo que se permite editar o no.

\section{Java}\label{java}

Java es un lenguaje de programación orientado a objetos, concurrente y de propósito general, y ha sido utilizado en el proyecto en las pruebas, con el WebDriver \footnote{Selenium WebDriver: Colección de enlaces (bindings) específicos para manejar el navegador} y entorno de pruebas de Selenium.

\section{Node.js}\label{Node.js}

Node es un intérprete Javascript en el lado del servidor. Permite construir aplicaciones muy escalables y  maneja miles de conexiones simultáneas en una sola máquina física.

En el proyecto ha sido utilizado para crear la REST API para manejar la base de datos. Se ha elegido por su simpleza, aunque cabe destacar que debido a que te permite manejar muchas opciones, es fácil equivocarse y realizar acciones que no se tenían pensadas.

\section{jQuery}\label{jquery}

jQuery podría se considerado una librería más que un framework. Está basada en JavaScript, y su función es simplificar la manera de interactuar con los documentos HTML, con los elementos DOM, manejar eventos, crear animaciones y permite el uso de AJAX \cite{jquerybib}.

En este proyecto se ha utilizado tanto sus funcionalidades de AJAX, como las de manejar eventos y las de manejar elementos DOM.

\section{Selenium}\label{selenium}

Selenium es un automatizador de los navegadores. Lo que quiere decir es que permite automatizar pruebas (acciones) directamente sobre una navegador a través de un WebDriver del propio navegador \cite{seleniumbib}.

Existen varias librerías de diferentes lenguajes para crear pruebas automatizadas. Pero para este proyecto se ha elegido en lenguaje Java.

\section{Webserver Stress Tool}\label{WebserverStressTool} 

Webserver Stress Tool es una aplicación de prueba HTTP-client/server diseñada para detectar problemas de rendimiento en un sitio web o servidor web \cite{webserverbib}.

En este proyecto se ha utilizado para realizar pruebas de rendimiento y estrés.

\section{Git}\label{Git}

Git es un software de control de versiones diseñado por Linus Torvalds. Se ha elegido porque es usado por el servicio web heroku, así como por GitHub, que es la plataforma en la que se aloja el repositorio del proyecto.

\section{ZenHub}\label{zenhub}

ZenHub es un plug in que permite añadir a GitHub características para mejorar la administración ágil del proyecto, así como mostrar estadísticas sobre el trabajo aplicado al proyecto.

\imagenResize{0.33}{img/zenhubboard}{Tablero ofrecido por ZenHub en GitGub}{zenhubboard}

%\ref{zenhubboard}

\section{\TeX{} y \LaTeX}\label{Latex}

\TeX{} es una herramienta para el manejo de \LaTeX{}, y este último es un sistema para la creación de documentos de texto de gran calidad tipográfica \cite{latexbib} \cite{texbib}.
Han sido utilizados para documentar ya que a pesar de que al principio se hace un poco difícil trabajar con ello, los resultados son excelentes.

\section{Zotero}\label{zotero}

Zotero es una herramienta y plug-in de Firefox que se ha utilizado para la administración de las referencias bibliográficas. Nos permite almacenar referencias a las páginas que se visitan a la hora de buscar información por la web. 

\section{Heroku}\label{heroku}

Heroku es una plataforma de servicio de computación en la nube que soporta distintos lenguajes de programación.
En el proyecto ha sido utilizado para sostener el servidor con Node.js con la Rest API. Se ha elegido Heroku por recomendación del tutor, así como sus facilidades para comenzar en ello, ya que ofrece servicios gratuitos y muy buenos tutoriales.

\section{Firebase}\label{firebase}

Firebase es una plataforma de desarrollo de aplicaciones móviles y web. Es un servicio de google y nos ofrece hosting, base de datos y almacenamiento, entre otras opciones. Ha sido elegido en este proyecto por su base de datos (JSON) no SQL en tiempo real, que permite trabajar con la base de datos de forma muy rápida y sencilla. El utilizar una base de datos JSON como AnyPlace (aplicación en la que se basa Geoindoor), si se quisiera aunar bases de datos, se conseguiría de una forma relativamente sencilla.


\section{HTML5}\label{HTML5}

HTML5 es un lenguaje de marcado que se ha utilizado para mostrar al usuario la información, así como para aprovechar algunas de sus caractrísticas para el procesamiento de información.

Se ha usado en el proyecto porque es muy utilizado, y es soportado por todos los dispositivos.

\section{CSS3}\label{CSS3}

CSS3 es un lenguaje de hojas de estilo en cascada utilizado para dar una impresión diferente al código HTML. Hay que decir que permite crear estilos de forma rápida y sencilla, pero al aumentar el número de clases y el anidamiento es muy difícil de mantener. 
 
\section{Bootstrap}\label{bootstrap}

Bootstrap es un framework para diseño de paginas web y aplicaciones web. Contiene elementos de diseño basados en HTML, CSS y JavaScript. Se ha utilizado para crear la interfaz y aplicarla estilos \cite{bootstrapbib}. 


\capitulo{5}{Aspectos relevantes del desarrollo del proyecto}

En el desarrollo de este proyecto han surgido varios obstáculos que han sido resueltos de diversas maneras. A continuación los explicaremos.

\section{Análisis e investigación de la aplicación }\label{analisis-investigacion}

El principal problema que me encontré fue entender la aplicación sobre la que trabajar y cómo hacerlo. La ausencia de documentación detallada dificultaba la labor, ya que no existía ningún documento donde se explicara el funcionamiento de la aplicación y cuáles son sus métodos y funciones. Con lo cual tuve que investigar y analizar la aplicación a conciencia hasta entenderla.

Otro gran problema que surgió, fue entender y crear el servidor que ofrecía AnyPlace, hasta que se decidió trabajar con la aplicación por separado, es decir, Architect por un lado y Viewer por otro y después unificarlos para crear la herramienta. 

\section{Viewer}\label{viewer}

En el proyecto en la parte del Viewer de la herramienta, un gran problema que ha surgido es el contenido mixto ya que muchas veces se ha mezclado contenido cifrado y contenido en plano.

Recordar que han surgido problemas a la hora de relacionar la base de datos de Geoindoor con el Viewer ya que en múltiples ocasiones el sistema de búsqueda anteriormente utilizado hacia devolver al servidor una gran cantidad de información que no era necesaria y ralentizaba las búsquedas (la búsqueda por id se hace menos costosa y más rápida).

\section{Architect}\label{Architect} 

En la herramienta en la parte de Architect, han surgido varios inconvenientes, pero los principales tienen que ver con el tema de procesado de datos, el dibujado sobre el plano y las llamadas a la REST API. En varias ocasiones el dibujado de las rutas no quedaba determinado por los pois, y no quedaba un resultado satisfactorio. A la hora de recoger los datos, muchos cambios introducidos por el usuario no causaban un resultado ``inmediato'', lo cual se solucionó gracias a JS y a AngularJS.

\section{Heroku}\label{heroku5}

El principal problema que tuve con ello es que no lo conocía, por lo que tuve que adaptarme y aprender su funcionamiento, aunque hay que decir que los primeros pasos los facilitan gracias a sus tutoriales.

He de señalar que el sistema de Heroku para dormir el servidor cuando no se utiliza hace, a veces, ralentizar las funcionalidades ofrecidas por Geoindoor, y al no tener una cuenta de pago los servicios que ofrece Heroku son limitados, pero no surge ninguna pega cuando está activo. 

Una vez que se aprendió cómo funcionaba heroku se comenzó a realizar la REST API cuyo principal problema fue entender como funcionaba Node.js y las llamadas POST Y GET en Node.js con express.

También es importante decir que con las llamadas al servidor surgieron problemas CORS (access-control-allow-origin),
que se han solucionado añadiendo las cabeceras de Access-Control-Allow-Origin. 

\section{Firebase}\label{firebase5}

Lo que sucedió con Firebase está relacionado con su base de datos JSON, cuya estructura de base de datos tuve que idear para que se pudiese conseguir una gran cantidad de información de calidad, evitando cantidad de información de poco valor.

La idea es utilizar keys iguales y añadir una palabra clave para utilizar esa clave como otra key. Es decir la key id2 es la key de un edificio y la key id2rutas tiene las rutas del edificio de key id2.


\section{Otros problemas}\label{otros}

En este apartado me gustaría hablar de uno de los problemas que mas roturas de cabeza me ha provocado y ha sido el CORS 'Access-Control-Allow-Origin', o control de acceso por origen. Cada vez que se hacía una solicitud desde un dominio diferente al oficial de AnyPlace, se bloqueaba la respuesta de esa solicitud, aun siendo perteneciente a la propia API que ofrece AnyPlace. Este obstáculo no solo aparecía en las peticiones en el Architect sino también en el Viewer, con lo cual la solución que le dí fue lanzar la herramienta de forma local por el puerto 9000 continuando las peticiones a la REST API de AnyPlace, evitando de esta manera el CORS, ya que en su servidor no tienen bloqueada esta acción.

El contenido mixto también me ha supuesto una dificultad, tanto en la parte de Architect, como la de Viewer, pero se solucionó al lanzar la herramienta de forma local con acceso a internet. 

Por otro lado, existieron otros muchos problemas que considero que no son relevantes. Aunque me gustaría destacar el problema que me encontré con el uso de jquery para las llamadas, sobre todo con el dataType, y los dynos de heroku a la hora de ofrecer servicios.
\capitulo{6}{Trabajos relacionados}

Los trabajos que estan relacionados con este son \href{https://www.google.es/maps}{Google Maps}, \href{https://anyplace.cs.ucy.ac.cy/}{AnyPlace} y \href{https://www.goindoor.co/}{goindoor}.

\section{Google Maps}\label{GoogleMaps}

Google Maps es la semilla del proyecto ya que AnyPlace se basa en ello.
Google Maps ofrece un API en JavaScript que ha sido ampliamente utilizado en este proyecto, la principal ventaja que tiene utilizar el API de Google Maps es el soporte y la atención, además la amplia comunidad que trabaja con Google Maps hace que se resuelvan la dudas fácilmente. 

La principal diferencia ente Google Maps y Geoindoor es que una Google Maps busca la geolocalización en extensiones amplias de terreno mientras que Geoindoor la busca en interirores, aunque sigue marcando rutas en largas distancias también.

\section{AnyPlace}\label{AnyPlace}

AnyPlace es la aplicación en la que se basa Geoindoor, con lo que las principales diferencias entre AnyPlace y Geoindoor son que Geoindoor además de cambios de estilo ofrece el agregado de rutas al edificio y su interactividad.

\section{goindoor}\label{goindoor}

goindoor es otra sistema utilizado para la geolocalización indoor y la principal diferencia entre Geoindoor y goindoor es que, goindoor utiliza bacons, o anclajes que son utilizados de sensores para obtener la geolocalización.

\capitulo{7}{Conclusiones y Líneas de trabajo futuras}

Para concluir puedo decir que el desarrollo ha sido difícil, sobre todo los primeros pasos que han consistido en el entendimiento del sistema y sus funciones.

A primera vista se podría pensar que la implementación de funcionalidades es sencilla pero no es tan sencillo una vez estas trabajando en ello. Sobre todo porque algunas llamadas de la API Rest de AnyPlace se encuentra capada.

En cuanto a los resultados, se ha conseguido integrar el agregado de rutas y su interacción.

En cuanto a las mejoras, se podría mejorar el viewer enlazando las rutas para su interacción. Ya se intentó pero debido a diversos problemas (CORS) no se ha conseguido para la entrega.

El problema del enlazado de las rutas con el Viewer se puede solucionar cogiendo la información a través de los controladores.

Este proyecto tiene un gran potencial y se podría utilizar en diversas situaciones, como la búsqueda de oficinas en infraestructuras publicas, rutas interactivas en centros comerciales o museos etc..  




\bibliographystyle{plainnat}
\bibliography{bibliografia}



\end{document}
