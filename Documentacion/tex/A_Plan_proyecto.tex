\apendice{Plan de Proyecto Software}

\section{Introducción}

La planificación es una parte muy importante del proyecto, ya que es la parte donde se estima el tiempo que se va consumir, el esfuerzo, y si ese proyecto va dar beneficios económicos.

En este anexo se va analizar los factores por los cuales es conveniente o no realizar un proyecto.
\begin{itemize}
\item
Tiempo consumido
\item
Viabilidad del proyecto
\item
Viabilidad económica
\item
Viabilidad legal
\end{itemize}
\section{Planificación temporal}

La planificación del tiempo es fundamental, ya que es muchas veces las prisas no son buenas consejeras, y tener mucho tiempo lleva a la procrastinación.

Al inicio del proyecto se planteo utilizar una metodologia ágil Scrum y utilizar ZenHub . Aunque no se ha seguido totalmente. Pero la idea era seguir los siguientes pasos.

\begin{itemize}
\item
  Aplicar una estrategia de desarrollo incremental a través de
  iteraciones y revisiones.
\item
  La duración media de los \emph{sprints} debe ser de una semana.
\item
  Al finalizar cada \emph{sprint} se entrega una parte del producto.
\item
  Se realiza reuniones de revisión al finalizar cada \emph{sprint} y se vuelve a planificar \emph{sprint}.
\item
  En la planificación del \emph{sprint} se genera una pila de tareas.
\item
  Estas tareas se estiman y priorizan en un tablero.
\item
  Para monitorizar el progreso del proyecto se utiliza gráficos
\end{itemize}

La idea era seguir estos pasos, pero debido a diversas dificultades no fue posible con lo cual se hicieron 2 Sprints,
\begin{itemize}
\item
El primer Sprint fue dedicado a familiarizarse con la aplicación, documentarse y conseguir planos.
\item
El segundo Sprint esta dedicado integramente al desarrollo integro de la aplicación.
\end{itemize}
 
\section{Estudio de viabilidad}

En cuanto al estudio de la viabilidad, se ha investigado y no hay mucha competencia en cuanto a geolocalización indoor basada en GPS.La mayoría de servicios de geolocalización indoor son de pago, y utilizan bacons para la localización. Lo que hace a Geoindoor uno de los sistemas más atractivos.


\subsection{Viabilidad económica}

En cuanto a la viabilidad económica,el desarrollo de la aplicación ha sido totalmente gratuito, una vez desarrollada la aplicación el único costo es el hosting. En este caso el hosting es gratuito 12 meses con lo cual, el primer año sería totalmente gratis, los siguientes años sería pagar 25.8 euros al año.

Si existe remuneración sería a partir de anuncios, pero no se contempla cobrar por el servicio. 

\subsection{Viabilidad legal}

En cuanto a la viabilidad legal no existe ningún problema siempre y cuando no se introduzca planos de forma ilegal o sin consentimiento, aunque la responsabilidad recaería sobre el individuo que sube el plano de forma ilegal o sin consentimiento.
