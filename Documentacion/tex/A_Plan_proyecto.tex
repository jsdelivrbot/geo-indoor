\apendice{Plan de Proyecto Software}

\section{Introducción}

La planificación es una parte muy importante del proyecto, ya que es la parte donde se estima el tiempo que se va consumir, el esfuerzo, y si ese proyecto va dar beneficios económicos.

He de decir que a la planificación de los tiempos y del trabajo se hace fundamental si se quiere ser productivo, existen factores no considerados como, que tipo de horas se ha invertido y con que lapsus de tiempo, que afectan a la productividad y a los tiempos. Existen muchos factores que pueden alterar estos parámetros, a continuación citaré algunos.

\begin{itemize}
\item
Horas de desarrollo después de una jornada de trabajo.
\item
Horas de desarrollo dispersas, por ejemplo, se desarrolla durante una hora se para durante horas y se vuelve a retomar, lo que hace que la productividad disminuya .
\item
Curva de aprendizaje, en este proyecto el aprendizaje ha sido fundamental y en muchas ocasiones costoso, si se junta este factor con los 2 anteriores la curva de aprendizaje crece con facilidad.
\item
Horas de desarrollo de poca calidad, es importante tener un lugar donde trabajar donde no se pierda la concentración y el hilo de forma continua.
\end{itemize}

En este anexo se va analizar los factores por los cuales es conveniente o no realizar un proyecto.
\begin{itemize}
\item
Planificación temporal
\item
Viabilidad del proyecto
\item
Viabilidad económica
\item
Viabilidad legal
\end{itemize}
\section{Planificación temporal}

La planificación del tiempo es fundamental, ya que es muchas veces las prisas no nos permiten razonar con serenidad, y tener mucho tiempo lleva a la procrastinación. Entonces lo ideal es llevar un trabajo constante y marcado, es decir crearse horarios de trabajo e intentar cumplirlos lo mejor posible.

Se ha utilizado una metodología ágil Scrum, basada en \emph{Sprints}, y ZenHub, para el seguimiento de issues y tareas. La plataforma GitHub y ZenHub nos permiten administrar el trabajo y las tareas de forma más dinámica y sencilla. Se ha intentado marcar \emph{Sprints} con una fecha final determinada, pero no ha sido posible debido a las dificultades a la hora de estimar el tiempo de trabajo y que no siempre existía la posibilidad de mantener un desarrollo en el tiempo, es decir, no siempre existían horas de calidad para aplicar al desarrollo.
De todas formas si que se han determinado \emph{Sprints} con un inicio y un final, los cuales detallaremos a continuación.
\\
\\
Estos son los pasos que se siguieron en el desarrollo: 

\begin{itemize}
\item
  Aplicar una estrategia de desarrollo incremental a través de
  iteraciones y revisiones.
\item
  La duración media de los \emph{Sprints} debe ser de una semana.
\item
  Al finalizar cada \emph{Sprint} se entrega una parte del producto.
\item
  Se realiza reuniones de revisión al finalizar cada \emph{Sprint} y se vuelve a planificar \emph{Sprint}.
\item
  En la planificación del \emph{Sprint} se genera una pila de tareas.
\item
  Estas tareas se estiman y priorizan en un tablero.
\item
  Para monitorizar el progreso del proyecto se utiliza gráficos
\end{itemize}

Se han seguido todos estos pasos, pero como ya se ha comentado anteriormente la finalización de los \emph{Sprints} no estaba determinada así que su final se tomará cuando se acabaron todos los issues pertenecientes al mismo. Los gráficos en muchas ocasiones no son totalmente representativos del trabajo ni el tiempo utilizado ya que en muchas ocasiones el tiempo invertido no era de calidad, o no se podía invertir tiempo.

\subsection{Sprint 0 (26/3/2017 - 3/4/2017)}\label{splrint0}
El \emph{Sprint} 0 fue el primer \emph{Sprint} que se creo y sirvió de toma de contacto, se afianzaron las ideas y se especificaron los objetivos del proyecto. Se puede decir que ha servido para afianzar ideas sobre el proyecto.
\imagenResize{0.4}{img/anexo/sprint0}{Gráfica Sprint 0}{imgsprint0}


\subsection{Sprint 1 (10/4/2017 - 10/8/2017)}\label{splrint1}
El \emph{Sprint} 1 ha sido uno de los más importantes ya que es donde se ha centrado la mayoría del aprendizaje y una gran cantidad de desarrollo, este \emph{Sprint} ha sido en el que se volcó por primera vez la herramienta Geoindoor en su conjunto, por lo que se crearon commits de gran tamaño.
\imagenResize{0.4}{img/anexo/sprint1}{Gráfica Sprint 1}{imgsprint1}


\subsection{Sprint 2 (10/8/2017 - 30/8/2017)}\label{splrint2}
Si miramos el \emph{Sprint} 2 podemos observar que la fase de aprendizaje está mas avanzada y que el desarrollo se hace menos costoso, pero también esta cargado de menos tareas que el anterior. En este proyecto cabe destacar que el aprendizaje no acaba siempre se encuentra algo nuevo.
\imagenResize{0.4}{img/anexo/sprint2}{Gráfica Sprint 2}{imgsprint2}


\subsection{Sprint 3 (25/9/2017 - 12/10/2017)}\label{splrint3}
El \emph{Sprint} 3 está más dedicado al desarrollo en la parte del viewer, con lo cual se tuvo que volver a investigar y a analizar esa parte de la herramienta para entender cual era la mejor manera para introducir modificaciones. Una ventaja es que entre Architect y Viewer hay un gran parecido, por lo tanto el esfuerzo no fue tanto como la primera vez. También decir que se tuvieron que hacer cambios para que las búsquedas de edificios fueran mas efectivas, lo que llevo un gran esfuerzo.
\imagenResize{0.4}{img/anexo/sprint3}{Gráfica Sprint 3}{imgsprint3}


\subsection{Sprint 4 (17/10/2017 - *)}\label{splrint3}
El \emph{Sprint} 4 es la última iteración antes de la entrega en ella se encuentran los últimos retoques a la herramienta así como, las pruebas realizadas y elementos dedicados a la documentación.
No tiene fecha final ya que abarca, como hemos dicho antes, los últimas pinceladas a la herramienta y la documentación. No encontramos gráfico por no tener fecha final.
 
\section{Estudio de viabilidad}

En cuanto al estudio de la viabilidad, se ha investigado y no hay mucha competencia en cuanto a geolocalización indoor basada en GPS.La mayoría de servicios de geolocalización indoor son de pago, y utilizan bacons para la localización. Lo que hace a Geoindoor uno de los sistemas más atractivos.


\subsection{Viabilidad económica}

En cuanto a la viabilidad económica,el desarrollo de la aplicación ha sido totalmente gratuito, una vez desarrollada la aplicación el único costo es el hosting. En este caso el hosting es gratuito 12 meses con lo cual, el primer año sería totalmente gratis, los siguientes años sería pagar 25.8 euros al año.

Si existe remuneración sería a partir de anuncios, pero no se contempla cobrar por el servicio. 

\subsection{Viabilidad legal}

En cuanto a la viabilidad legal no existe ningún problema siempre y cuando no se introduzca planos de forma ilegal o sin consentimiento, aunque la responsabilidad recaería sobre el individuo que sube el plano de forma ilegal o sin consentimiento.
