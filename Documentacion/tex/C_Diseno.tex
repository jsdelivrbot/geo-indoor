\apendice{Especificación de diseño}

\section{Introducción}
En este apartado, nos detendremos a detallar la estructura de datos, los procedimientos y la arquitectura, que sigue la herramienta Geoindoor.
\section{Diseño de datos}

Este proyecto trabaja con una base de datos no sql por lo tanto la manera en la que están estructurados los datos, es un poco diferente a la que estamos acostumbrados con bases de datos sql.
Los datos en vez de estar estructurados en tablas (BD sql), se estructuran en documentos\footnote{Un documento es un conjunto de entidades de bases de datos que se agrupan en una entidad}. 

Además Geoindoor utiliza el servicio web de bases de datos de Firebase, por lo que la BD está formada por un único documento. Se usa JSON para el almacenamiento de datos, lo que nos lleva a utilizar un sistema de \textit{clave-valor} para la ordenación.

En conclusión podríamos resumir las características de la BD de Geoindoor en 5 puntos:

\begin{itemize}
	\item Base de datos no sql.
	\item Documentos en vez de entidades relacionales.
	\item Estructura de un único documento.
	\item Almacenamiento de datos a través \textit{clave-valor}
	\item Utilización de JSON para almacenamiento de datos
\end{itemize} 

Entonces la manera de analizar el diseño de datos no se puede realizar desde el punto de vista relacional. De todas formas, a continuación se muestran diagramas que permiten entender mejor el diseño y estructura de datos de Geoindoor.

\imagenResize{0.5}{img/DiagramaRelacionEntidades}{Diagrama de relacion de entidades}{DiagramaRelacionEntidades}

Estos diagramas se leen de izquierda a derecha, para saber que entidad contiene a cual. Las entidades moradas contienen otras entidades, mientras que las azules guardan la relación \textit{clave-valor}, donde se almacenan realmente los datos. Todas las entidades se encuentran en un mismo documento (Geoindoor DB).

La relación virtual indica que hay una relación lógica entre esas entidades, es decir, hay relación de datos entre entidades. En este caso la entidad ``id1 - id..n'', que identifica a los edificios, está relacionada con ``id1rutas - id..nrutas'', que identifica las rutas de cada edificio. De ésta manera podríamos decir que la relación entre ``id1 - id..n'' y ``id1rutas - id..nrutas' es de 1 - *.

\newpage
\textbf{ Geoindoor DB entidades:}
\begin{itemize}
	\item \textbf{Contador}
	\begin{itemize}
		\item Clave 1: id
	\end{itemize}
	\item \textbf{Emails}
	\begin{itemize}
		\item \textbf{e1 - e..n}: identificadores de email.
		\begin{itemize}
			\item Clave 1: contrasena
			\item Clave 2: email 
		\end{itemize}
	\end{itemize}
	\item \textbf{Edificios}
	\begin{itemize}
		\item \textbf{id1 - id..n}: identificadores de edificio.
		\begin{itemize}
			\item Clave 1: idmail 
			\item Clave 2: nombre
		\end{itemize}
		\item \textbf{Rutas}
		\begin{itemize}
			\item \textbf{id1rutas - id..nrutas}: identificadores de ruta de edificio.
			\begin{itemize}
				\item \textbf{r1 - r..rn}: identificadores de ruta. \\
				\hspace{0.5cm} Clave 1: nombre \\
				\hspace{0.5cm} Clave 2: rutas\\
				\hspace{0.5cm} Clave 3: puntos \\
			\end{itemize}
		\end{itemize}
	\end{itemize}
\end{itemize}




\section{Diseño procedimental}

En el diseño procedimental explica cómo interaccionan las distintas partes del algoritmo.
A continuación se muestran unos diagramas de secuencia que describen el comportamiento de los algoritmos utilizados para el correcto funcionamiento de la herramienta Geoindoor. No se exponen todos los diagramas de secuencia resultantes de el análisis de los algoritmos, por que no son relevantes para el proyecto Geoindoor.

Para los diagramas de la parte de Architect de Geoindoor, el usuario ya está logeado una vez comienza la acción.
\newpage
\imagenResize{0.5}{img/DiagramasSecuencia/DSCrearEdificio}{Diagrama de secuencia de la creación de un edificio}{DSCrearEdificio}

Para que el siguiente diagrama tenga sentido hay que ubicarle en un contexto. Debe haber un edificio creado y con al menos dos pois.
\newpage
\imagenResize{0.3}{img/DiagramasSecuencia/DSCrearRuta}{Diagrama de secuencia de la creación de una ruta predefinida}{DSCrearRuta}

Para el siguiente diagrama el contexto cambia. Ahora deber haber un edificio con al menos una ruta predefinida.

\imagenResize{0.5}{img/DiagramasSecuencia/DSDibujadoRuta}{Diagrama de secuencia del dibujado de una ruta predefinida al seleccionarla}{DSDibujadoRuta}

\newpage
El siguiente diagrama tiene el mismo contexto que el anterior.

\imagenResize{0.5}{img/DiagramasSecuencia/DSBorrarRuta}{Diagrama de secuencia del borrado de una ruta}{DSBorrarRuta}

\newpage
En este diagrama, nos encontramos en Geoindoor Viewer, en un edificio con al menos una ruta predefinida.

\imagenResize{0.5}{img/DiagramasSecuencia/DSDibujarRutaV}{Diagrama de secuencia del dibujado de una ruta (Viewer)}{DSDibujarRutaV}
 
\section{Diseño arquitectónico}

En este apartado se hablará de cómo esta diseñado el sistema Geoindoor a nivel de su arquitectura de software.

El patrón que se ha elegido para implementar y que en gran parte ha sido impuesto por la herramienta en la que se ha basado Geoindoor (AnyPlace), es el patrón Modelo-Vista-Controlador (MVC).

El patrón MVC busca que la interfaz de usuario sea totalmente independiente de la lógica del sistema, pero con Geoindoor y AnyPlace, esta independencia, en ciertos aspectos se vuelve muy difícil de conseguir, ya que JavaScript y AngularJS dependen en gran medida de la interfaz HTML, lo que en varias ocasiones ha hecho que haya una relación de dependencia entre lógica e interfaz.

\subsection{Modelo Vista Controlador}

Este patrón se basa en separar los datos y la lógica del sistema de la interfaz del usuario. Para tal finalidad el MVC propone la construcción de 3 componentes: \textbf{modelo}, \textbf{vista} y \textbf{controlador} \cite{mvcwiki}.

\begin{itemize}
	\item \textbf{Modelo:} es el encargado de mantener los datos, es decir, es el que se encarga de la representación y gestión (actualizaciones, borrados, consultas, creaciones) de la información. Envia a la \textit{Vista} la información que le solicita el \textit{Controlador}.
	\item \textbf{Controlador:} Es el encargado de controlar la interacción entre la \textit{Vista} y el \textit{Modelo}. Responde a eventos como acciones del usuario, y transmite las peticiones al modelo.
	\item \textbf{Vista:} es el componente encargado de mostrar la información recibida del \textit{Modelo} al usuario y transmitir las peticiones del usuario al \textit{Controlador}.
\end{itemize}

\imagenResize{1}{img/MVC/ModelViewControllerDiagram}{Diagrama Modelo-Vista-Controlador \cite{mvcwiki}}{ModelViewControllerDiagram}

La abstracción del funcionamiento de un sistema Modelo-Vista-Controlador, puede ser representada de la siguiente forma.

\imagenResize{0.5}{img/MVC/angularjsmvc}{Diagrama de flujo Modelo-Vista-Controlador \cite{mvcangular}}{angularjsmvc}

\subsection{Modelo-Vista-Controlador en Geoindoor}

En cuanto a cómo está implementado el Modelo-Vista-Controlador en Geoindoor podemos categorizar cada elemento participante, en cada uno de los estratos del MVC.

\begin{itemize}
	\item \textbf{Modelo:} en el modelo podemos introducir el servicio de base de datos de Firebase (Geoindoor), la REST API creada en Heroku (Geoindoor), la BD de AnyPlace y la REST API de AnyPlace. Ambas API REST nos permiten interactuar con las respectivas bases de datos.
	\begin{itemize}
		\item BD Geoindoor
		\item REST API Geoindoor 
		\item BD AnyPlace
		\item REST API AnyPlace 
	\end{itemize}
	\item \textbf{Controlador:} en el componente de controladores podemos añadir el fichero ``\textit{app.js}''\footnote{app.js: es el fichero en el que se declara cada aplicación, y está basada en un modulo, a partir de el cual surgen los controladores.}, todos los ficheros de la carpeta ``\textit{controllers}'' y los ficheros de la carpeta  ``\textit{scripts}'' que son utilidades, usadas por los controladores.
	\item \textbf{Vista:} en la vista añadiremos el fichero ``\textit{index.html}''. Bootstrap, CSS y demás elementos para el estilo se podrían introducir en componentes que forman la \textit{Vista}.
\end{itemize}

\imagenResize{0.3}{img/MVC/MVCArchitect}{Elementos integrantes del Modelo-Vista-Controlador en Architect}{MVCArchitect}

\imagenResize{0.3}{img/MVC/MVCViewer}{Elementos integrantes del Modelo-Vista-Controlador en Viewer}{MVCViewer}

\subsection{Navegabilidad}

En cuanto a la navegabilidad del sistema, es muy simple, una vez arrancada la herramienta o sistema Geoindoor, aparecerá una pantalla de inicio, dónde encontramos dos botones que pemitirán al usuario elegir si trabajar con la herramienta Architect o Viewer, de Geoindoor. 

\imagenResize{0.6}{img/MVC/Nav}{Diagrama de navegabilidad de Geoindoor}{Nav}

No se considera la fácil navegación (idas y vueltas intermitentes) entre la herramienta Architect y Viewer, ya que para el usuario no tiene ningún sentido esa navegación. 